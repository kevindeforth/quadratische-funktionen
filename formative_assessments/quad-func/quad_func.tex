\documentclass[12pt]{article}
%\usepackage{pgfkeys}
\usepackage[widelayout,sf,ngerman, solution, copyright, hyperref, pagecount]{custom23}
\usepackage{func_helper}
%\stefancopyright=true
\newcommand{\docName}{Quadratische Funktionen}
%\newcommand{\docVersion}{Version 0.0.0}
\newcommand{\klasse}{Kantonsschule Wettingen, Klasse G1D}

\fancyhead[L]{\klasse}
\fancyhead[R]{\today}%, \docVersion}
\renewcommand{\headrulewidth}{0.4pt} % Line under the header
\usepackage{wrapfig}

\usepackage{tabularx}
\newcolumntype{Y}{>{\centering\arraybackslash}X}
\renewcommand\tabularxcolumn[1]{m{#1}}% for vertical centering text in X column

\usepackage{setspace}
\usepackage{subcaption}
\setboolean{solution}{true}

\usepackage{soul}

\begin{document}
\begin{center}
\Large \textbf{Formatives Assessment Quadratische Funktionen}\\[1em]
\footnotesize
\end{center}
{\footnotesize
\subsubsection*{Hinweis}
Dieses Assessment hat rein formativen Charakter. Das Resultat nimmt keinen Einfluss auf Ihre Zeugnisnote, sondern dient einzig dazu, Lerninhalte zu identifizieren, die noch mehr Ihrer Aufmerksamkeit benötigen.
\begin{itemize}
\item $\mathbb{R}$ ist die Menge der reellen Zahlen
\item $\mathbb{R} \setminus \{0\}$ ist die Menge der reellen Zahlen ohne die Zahl $0$.
\item Das Symbol $\forall$ bedeutet \emph{``für alle''}, oder \emph{``für jedes''}. Der Ausdruck $\forall x \in A$ bedeutet: \emph{``Für jedes Element $x$ der Menge $A$''}.
\end{itemize}
}
\subsection*{Aufgabe 1}

\realFun{f}{t}{s(t+3)^2 - 5}
Es sei $s \in \mathbb{R} \setminus \{0\}$ ein reeller Parameter und $f$ definiert wie folgt. \printFunDef{f}
\begin{enumerate}[label=\alph*)]
\item Vervollständigen Sie den Lückentext mit den Symbolen $<,>, \leqslant, \geqslant,0,2$ und den Worten \texttt{Maximum}, \texttt{Minimum}, \texttt{oben} und \texttt{unten}.
\begin{quote} \doublespacing
Wenn $s \; \rule{0.5cm}{0.5pt} \; 0$ gilt, dann ist der Funktionsgraph von $f$ nach \rule{2cm}{0.5pt} geöffnet. In diesem Fall gilt $f(t)\; \rule{0.5cm}{0.5pt} \; -5, \; \forall t \in \Reals$ und die Funktion $f$ weist ein \rule{2cm}{0.5pt} auf. Die Funktion hat in diesem Fall genau \rule{0.5cm}{0.5pt} Nullstellen.


Wenn $s \; \rule{0.5cm}{0.5pt} \; 0$ gilt, dann ist der Funktionsgraph von $f$ nach \rule{2cm}{0.5pt} geöffnet. In diesem Fall gilt $f(t)\; \rule{0.5cm}{0.5pt} \; -5, \; \forall t \in \Reals$ und die Funktion $f$ weist ein \rule{2cm}{0.5pt} auf. Die Funktion hat in diesem Fall genau \rule{0.5cm}{0.5pt} Nullstellen.
\end{quote}
\item Bestimmen Sie den Ordinatenabschnitt von $f$
%\item Welcher Funktionsterm entsteht, wenn Sie den Funktionsgraph von $f$ um $2$ Einheiten nach oben und um $3$ Einheiten nach rechts verschieben?
\end{enumerate}


%\subsection*{Aufgabe 2}
%\realFun{g}{t}{4(t-3)^2 + d_g}
%Es seien $d_g \in \mathbb{R}$ ein reeller Parameter und die Funktion $g$ definiert wie folgt. \printFunDef{g}
%Bestimmen Sie die Menge der Elemente, für welche $g$ genau
%\begin{enumerate}[label=\alph*)]
%\item keine Nullstelle hat;
%\item eine Nullstelle hat;
%\item zwei Nullstellen hat.
%\end{enumerate}

\subsection*{Aufgabe 2}
\realFun{h}{x}{x^2 - 2x + 1}% nutzen, um binomische Formel zu repetieren.
\realFun{u}{x}{-3x^2 - 2x + 1}
Es seien $u$ und $h$ zwei Funktionen wie folgt definiert
\begin{IEEEeqnarray*}{rCrClXrCrCl}
\getRelation{h} & \qquad &\getRelation{u}\\
\getMap{h} & \qquad & \getMap{u}.
\end{IEEEeqnarray*}

\begin{enumerate}[label=\alph*)]
\item Bestimmen Sie den Ordinatenabschnitt von $h$ und $u$
\item Bestimmen Sie den Scheitelpunkt von $h$ und $u$\\[5cm]
\end{enumerate}

\subsection*{Aufgabe 3}
\realFun{v}{x}{}
\realFun{g}{x}{(x-1)^1 + 1}
Es sei $\getRelation*{v}$ eine quadratische Funktion mit Nullstellen in $-4$ und $-8$. Der Funktionswert von $v$ im Scheitelpunkt ist $2$. Bestimmen Sie den Funktionsterm von $v$.
%Wenn der Parameter $d_g$ \rule{2cm}{0.5pt} ist, hat $g$ genau \rule{2cm}{0.5pt} Nullstellen. Wenn der Parameter $d_g$  \rule{2cm}{0.5pt} ist, hat $g$ genau \rule{2cm}{0.5pt} Nullstellen. Und wenn $d_g$  \rule{2cm}{0.5pt} ist, dann hat $g$ \rule{2cm}{0.5pt} Nullstellen.
%\begin{enumerate}[label=\alph*)]
%\item 
%\item Welche Werte kann Parameter $d_g$ annehmen, damit die Funktion $g$: \begin{enumerate}
%\item keine Nullstelle hat?
%\item eine Nullstelle hat?
%\item zwei Nullstellen hat?
%\end{enumerate} 
%\end{enumerate}



%\subsection*{Aufgabe 3}
%\realFun{f}{t}{at^2 + bt + c}
%Es sei die Funktion $f$ definiert wie folgt.
%\printFunDef{f}
%\begin{enumerate}
%\item Der Funktionsgraph
%\end{enumerate}
%
%Es seien $a,b,c,d$ und $e$ reelle Zahlen, wobei $a\neq 0$ gilt und die Funktionen $f$ und $g$ definiert sind als
%\begin{IEEEeqnarray*}{rCrClXrCrCl}
%\getRelation{f} & \qquad &\getRelation{g}\\
%\getMap{f} & \qquad & \getMap{g}.
%\end{IEEEeqnarray*}
%
%\begin{enumerate}[label=\alph*)]
%\item Bestimmen Sie den Ordinatenabschnitt von $f$ und $g$.\\[4cm]
%\item Bestimmen Sie den Scheitelpunkt von $f$ und $g$.\\[4cm]
%\item Bestimmen Sie für jede der folgenden Aussagen, ob diese richtig oder falsch sind: \begin{enumerate}[label=\roman*)]
%\item Es gilt $f(-\frac{b}{2} + t) = f(-\frac{b}{2}-t),$ für alle $t\in \Reals$.
%\item Es gilt $g(t+d) = g(t-d),$ für alle $t\in \Reals$.
%\item Die Funktion $f$ hat ein Minimum genau dann, wenn $a$ negativ ist.
%\item Die Funktion $g$ ist offensichtlich immer positiv, wenn $a$ positiv ist.
%\end{enumerate}
%\item Korrigieren Sie folgende Aussage: \verb=Der Scheitelpunkt entspricht dem Punkt, in dem die Funktion ihr Maximum erreicht.=
%\end{enumerate}
%
%
%Vervollständigen Sie folgenden Text:
%{
%\onehalfspacing
%Die Funktion $g$ ist eine \rule{2cm}{0.5pt} Funktion mit Scheitelpunkt in \rule{2cm}{0.5pt}. Der Scheitelpunkt entspricht dem Punkt, in dem die Funktion ihr \rule{2cm}{0.5pt} erreicht, wenn $a$ positiv ist. Wenn $a$ negativ ist, erreicht die Funktion im Scheitelpunkt ihr \rule{2cm}{0.5pt}.
%
%Der Funktionswert im Scheitelpunkt entspricht dem Extremwert der Funktion. 
%
%
%Die Funktion $g$ ist symmetrisch um 
%Eine quadratische Funktion 
%}
%- Bestimmen des Scheitelpunktes:
%   - wenn in scheitelpunktform
%   - wenn in der allgemeinen Form
%   - anhand von 3 
%- Bestimmen der Extremwerte (min / max)
%- Bestimmen der Symmetrieachse
%
%Bestimmen Sie für die folgenden Funktionen den Scheitelpunkt
%sowie die Extremwerte
% minimum, maximum bestimmen
\label{lastpage}

\end{document}