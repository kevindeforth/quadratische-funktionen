\documentclass[12pt]{article}
\usepackage{pgfkeys}
\usepackage[widelayout,sf,ngerman, solution, copyright, hyperref, pagecount]{custom23}
%\stefancopyright=true
\newcommand{\docName}{Quadratische Funktionen}
%\newcommand{\docVersion}{Version 0.0.0}
\newcommand{\klasse}{Kantonsschule Wettingen, Klasse G1D}

\fancyhead[L]{\klasse}
\fancyhead[R]{\today}%, \docVersion}
\renewcommand{\headrulewidth}{0.4pt} % Line under the header
\usepackage{wrapfig}

\usepackage{tabularx}
\newcolumntype{Y}{>{\centering\arraybackslash}X}
\renewcommand\tabularxcolumn[1]{m{#1}}% for vertical centering text in X column


\usepackage{subcaption}
\setboolean{solution}{true}

\usepackage{soul}

\begin{document}
\begin{center}
\LARGE \textbf{Formatives Assessment, Funktionsbegriff}\\[1em]
\small
Dieses Assessment hat rein formativen Charakter. Das Resultat nimmt keinen Einfluss auf Ihre Zeugnisnote.
Das Resultat dient einzig dazu, Lerninhalte zu identifizieren, die noch mehr Ihrer Aufmerksamkeit benötigen.
\end{center}

\subsubsection*{Hinweis}
\begin{itemize}
\item $\mathbb{R}$ ist die Menge der reellen Zahlen
\item $\mathbb{R}_{+}$ ist die Menge der positiven reellen Zahlen (grösser als null)
\item $\mathbb{R}_{-}$ ist die Menge der negativen reellen Zahlen (kleiner als null)
\item $\mathbb{R}_{\geqslant 0}$ ist die Menge der nicht-negativen reellen Zahlen (grösser gleich null)
\item $\mathbb{R}_{\leqslant 0}$ ist die Menge der nicht-positiven reellen Zahlen (kleiner gleich null)
\end{itemize}
\subsection*{Aufgabe 1}
Was ist eine Funktion?\\
\begin{solution}

Eine Funktion ist eine Beziehung zwischen zwei Mengen: der Definitionsmenge und der Zielmenge.
Eine Funktion ordnet jedem Element der Definitionsmenge genau ein Element der Zielmenge zu.

\end{solution}
\newpage
\subsection*{Aufgabe 2}
Es seien folgende Funktionen:
\begin{IEEEeqnarray*}{rCrClCrCrClCrCrClCrCrClCrCrCl}
f &:& \mathbb{R}_{-} &\rightarrow & \mathbb{R} 
& \quad \quad &
g &:&\mathbb{R} &\rightarrow & \mathbb{R}
& \quad \quad &
h &:&\mathbb{R} &\rightarrow & \mathbb{R}
& \quad \quad &
\bigstar &:&\mathbb{R}_{+} &\rightarrow & \mathbb{R}_{+}
& \quad \quad &
\square &:&\mathbb{R} &\rightarrow & \mathbb{R}_{+}
\\
&& x &\mapsto &x^2 
&&
&& y &\mapsto &y+1 
&&
&& z &\mapsto &0
&&
&& t &\mapsto &\frac{1}{t}
&&
&& \blacktriangle &\mapsto &\blacktriangle ^2
\end{IEEEeqnarray*}

\noindent\hl{Leider ist mir hier ein Fehler unterlaufen. Die Funktion $\square$ ist schlecht definiert, da der Funktionswert im Punkt $0$ nicht in der Zielmenge enthalten ist.}

\begin{enumerate}[label=\alph*)]
\item Was ist der Funktionswert von $\square$ im Punkt $5$?\\
\begin{solution} Der Funktionswert von $\square$ im Punkt $5$ lässt sich ermitteln, indem man $5$ als Funktionsargument einsetzt: $\square(5) = 5^2 = 25$.
\end{solution}
\item Was ist die Definitionsmenge der Funktion $f$?\\
\begin{solution}
Die Menge der negativen reellen Zahlen.
\end{solution}
\item Was ist die Zielmenge der Funktion $f$?\\
\begin{solution}
Die Menge der reellen Zahlen.
\end{solution}
\item Klassifizieren Sie jede der Funktionen als \emph{(nicht-)injektiv}, \emph{(nicht-)surjektiv}, \emph{(nicht)-bijektiv}.\\

\begin{tabular}{|c|c|c|c|}
\toprule
Funktion & injektiv & surjektiv & bijektiv\\
& ja / nein & ja / nein & ja / nein\\
\midrule
$f$ & ja & nein & nein\\
\hline
$g$ & ja & ja & ja\\
\hline
$h$ & nein & nein & nein\\
\hline
$\bigstar$ & ja & ja & ja\\
\hline
$\square$ & \multicolumn{3}{c|}{schlecht definiert}\\
\bottomrule
\end{tabular}
\begin{solution}
\begin{enumerate}[label=\alph*)]
\item[$f:$] Die Funktion ist injektiv, da für alle $x_1, x_2 \in \mathbb{R}_{-}$ gilt:
\begin{IEEEeqnarray*}{rCl}
f(x_1) = f(x_2) &\iff & x_1^2 = x_2^2\\
&\iff & x_1 = \pm x_2\\
&\iff & x_1 = x_2 \text{ (da $x_1$ und $x_2$ beide negativ sind)}
\end{IEEEeqnarray*}
Die Funktion ist nicht surjektiv, da kein Element der Definitionsmenge dem Element $0$ der Zielmenge zugeordnet wird:
\begin{IEEEeqnarray*}{rCl}
f(x) = 0 & \iff & x^2 = 0\\
& \iff & x = 0, \text{(aber } 0 \not \in \mathbb{R}_{-}\text{)}
\end{IEEEeqnarray*}
Da die Funktion nicht surjektiv ist, kann sie nicht bijektiv sein.
\item[$g:$] Die Funktion $g$ ist injektiv, denn es gilt für alle $y_1, y_2 \in \mathbb{R}:$
\begin{IEEEeqnarray*}{rCl}
g(y_1) = g(y_2) & \iff & y_1 + 1 = y_2 + 1\\
& \iff & y_1 = y_2.
\end{IEEEeqnarray*}
Die Funktion $g$ ist ausserdem surjektiv, denn für jedes Element $z$ der Zielmenge $\mathbb{R}$, gibt es das Element $z-1$ der Definitionsmenge, dessen Funktionswert $z$ ergibt:
\begin{IEEEeqnarray*}{rCcCl}
 g(z-1) &=& (z-1) + 1 &=& z, \forall \; z \in \mathbb{R}
\end{IEEEeqnarray*}
Da die Funktion injektiv und surjektiv ist, ist sie bijektiv.
\item[$h$] Die Funktion $h$ ist nicht injektiv, da zum Beispiel $1$ und $2$ in der Definitionsmenge liegen und beide Elemente den gleichen Funktionswert annehmen:
\begin{IEEEeqnarray*}{rCl}
h(1) &=& 0\\
h(2) &=& 0
\end{IEEEeqnarray*}
Da $h$ nicht injektiv ist, kann $h$ nicht bijektiv sein.

$h$ ist auch nicht surjektiv, da zum Beispiel $1$ in der Zielmenge liegt, jedoch nie als Funktionswert angenommen wird.

\item[$\bigstar$] Die Funktion $\bigstar$ ist injektiv, denn es gilt für alle $t_1, t_2 \in \mathbb{R}_{+}$:
\begin{IEEEeqnarray*}{rCl}
\bigstar(t_1) = \bigstar(t_2) & \iff & \frac{1}{t_1} = \frac{1}{t_2}\\
& \iff & t_1 = t_2\\
\end{IEEEeqnarray*}

Die Funktion $\bigstar$ ist ausserdem surjektiv, denn für jedes Element $y$ der Zielmenge, existiert das Element $\frac{1}{y}$ der Definitionsmenge, dessen Funktionswert gleich $y$ ist:
\begin{IEEEeqnarray*}{rClR}
\bigstar(\frac{1}{y}) &=& \frac{1}{\frac{1}{y}}&\\
& = & \frac{y}{y\frac{1}{y}} & \text{ (Zähler und Nenner mit $y$ multiplizieren)}\\
& = & \frac{y}{\frac{y}{y}}\\
& = & \frac{y}{1}\\
&=& y
\end{IEEEeqnarray*}
Da die Funktion injektiv und surjektiv ist, ist sie bijektiv.
\item[$\square$] Die Funktion $\square$ ist leider schlecht definiert. Wenn $\square$ wie folgt definiert wäre:
\begin{IEEEeqnarray*}{rCrCl}
\square &:& \mathbb{R} &\rightarrow& \mathbb{R}_{\geqslant 0}\\
&& \blacktriangle &\mapsto & \blacktriangle^2,
\end{IEEEeqnarray*}
dann wäre die Funktion nicht injektiv, denn wir finden $\square(1) = 1 = \square(-1)$.
Da die Funktion nicht injektiv ist, kann sie nicht bijektiv sein.

Die Funktion wäre dann aber surjektiv, denn für jedes Element $y$ der Zielmenge, existiert das Element $\sqrt{y}$ der Definitionsmenge, dessen Funktionswert gleich $y$ gibt:
\begin{IEEEeqnarray*}{rCl}
\square(\sqrt{y}) = (\sqrt{y})^2 = y.
\end{IEEEeqnarray*}
\label{lastpage}
\end{enumerate}
\end{solution}
\end{enumerate}
\end{document}