\documentclass[12pt]{article}
 
\usepackage[margin=1in]{geometry}
\usepackage{amsmath,amsthm,amssymb}
\usepackage{enumitem}
\usepackage[utf8]{inputenc}
\usepackage{enumitem} 
\usepackage[flushleft]{threeparttable} 
\usepackage{multirow}
\usepackage{longtable}
\usepackage{tikz}
\usepackage{pgfplots}
\usepackage{dynkin-diagrams}
\usetikzlibrary{backgrounds}
\usepackage{multicol}
\usepackage{tikz_helper24}
 
\newcommand{\N}{\mathbb{N}}
\newcommand{\Z}{\mathbb{Z}}
\newcommand{\R}{\mathbb{R}}
\DeclareMathOperator{\diag}{diag}
\DeclareMathOperator{\codim}{codim}
\newcommand{\GO}{\operatorname{GO_{n}(\mathbb{R})}}
\newtheorem{prop}{Proposition}[section]
\newtheorem{ex}{Exercise}[section]
\newtheorem{defin}{Definition}[section]
\newtheorem{rem}{Remark}[section]
\newtheorem{sol}[prop]{Solution}
\newtheorem{lem}[prop]{Lemma}
\newtheorem{cor}[prop]{Corollary}
\newtheorem{theorem}{Theorem}[section]
\newtheorem{formula}{Formula}[section]


\DeclareMathOperator{\GL}{GL}
\DeclareMathOperator{\SL}{SL}
\DeclareMathOperator{\Sp}{Sp}
\DeclareMathOperator{\Aut}{Aut}
\DeclareMathOperator{\Lie}{Lie}
\DeclareMathOperator{\rank}{rank}
\DeclareMathOperator{\ZG}{Z}
\DeclareMathOperator{\ZM}{Z(M)}
\DeclareMathOperator{\CG}{C}
\DeclareMathOperator{\NRG}{N}
\DeclareMathOperator{\CLG}{C_{\Lie(G)}}
\DeclareMathOperator{\ZC}{Z(C_{G}}
\DeclareMathOperator{\SW}{S^{2}}
\DeclareMathOperator{\id}{id}
\DeclareMathOperator{\ord}{ord}
\DeclareMathOperator{\chr}{char}
\DeclareMathOperator{\range}{range}

\newlist{legal}{enumerate}{10}
\setlist[legal]{label*=\emph{\arabic*.}}


%\documentclass{article}
%\usepackage{lipsum} % This package generates filler text.
\usepackage{fancyhdr}
\pagestyle{fancy}
\fancyhf{} % Clear default header and footer
\fancyfoot[L]{\copyright\ Dr.~A.~M.~Deforth} % Left footer
\fancyfoot[C]{} % Center footer (empty in this case)
\fancyfoot[R]{\thepage} % Right footer
\renewcommand{\headrulewidth}{0pt} % Optional: removes the line in the header
\renewcommand{\footrulewidth}{0.4pt} % Optional: adds a line above the footer (adjust thickness as needed)


\begin{document}
\title{A little bit of Arithmetics}
\maketitle
\begin{center}
\copyright\ Dr.~A.~M.~Deforth
\end{center}
\section{Generalities}
\subsection{Addition}
The result of addition is called the \underline{sum}. Addition has the following properties:
\begin{description}
\item[Neutral element]: $\text{for any } x\in \mathbb{R} \text{ we have } x+0=0+x=x$.
\item[Associativity]: $\text{for any } x,y,z\in \mathbb{R} \text{ we have } x+(y+z)=(x+y)+z$.
\item[Commutativity]: $\text{for any } x,y\in \mathbb{R} \text{ we have } x+y=y+x$.
\end{description}


\subsection{Multiplication}
The result of multiplication is called the \underline{product}. Multiplication has the following properties:
\begin{description}
\item[Neutral element]: $\text{for any } x\in \mathbb{R} \text{ we have } x\cdot 1=1\cdot x=x$.
\item[Associativity]: $\text{for any } x,y,z\in \mathbb{R} \text{ we have } x\cdot (y\cdot z)=(x\cdot y) \cdot z$.
\item[Commutativity]: $\text{for any } x,y\in \mathbb{R} \text{ we have } x\cdot y=y\cdot x$.
\item[Distributivity with respect to addition]: $\text{for any } x,y,z\in \mathbb{R} \text{ we have }$ $$x\cdot (y+z)=x\cdot y+x\cdot z.$$
\end{description}

\section{Evaluating expressions}

A \underline{mathematical expression} is any meaningful combination of numbers, letters, operation symbols (such as $+,-,\cdot,\div,\sqrt{}$) and grouping symbols (such as parentheses, brackets, fraction bars and extended square root symbol). 

\textit{Example}: The expression: 
$$\frac{x+y}{x-y}$$
indicates a fraction whose numerator is the sum of two unspecified numbers, $x\in \mathbb{R}$ and $y\in \mathbb{R}$, and whose denominator is their difference. 

A \underline{variable} is a symbol that represents an unspecified number that can take various different values. They are represented by letters, conventionally letters at the end of the alphabet: $x,y,z, t$, etc. Expressions can be \underline{evaluated} (assigned a numerical value) if numerical values are assigned to all the variables appearing in the expression.

\textit{Example}: We want to evaluate $\frac{x+y}{x-y}$ for $\textcolor{red}{x=2}$ and $\textcolor{blue}{y=-6}$. For this, we replace each variable by its assigned numerical value:
$$\frac{\textcolor{red}{2}+\textcolor{blue}{(-6)}}{\textcolor{red}{2}-\textcolor{blue}{(-6)}}=\frac{-4}{8}=\frac{1}{2}.$$

Assigning different values to the variables in an expression usually changes the value of the expression.

\textit{Example}: This time, we want to evaluate $\frac{x+y}{x-y}$ for $\textcolor{red}{x=-3}$ and $\textcolor{blue}{y=3}$. As before, we replace each variable by its assigned numerical value:
$$\frac{\textcolor{red}{(-3)}+\textcolor{blue}{3}}{\textcolor{red}{(-3)}-\textcolor{blue}{3}}=\frac{0}{-6}=0.$$

\begin{ex}\label{Ex1}
Evaluate:
\begin{multicols}{2}
\begin{enumerate}
\item[a)] $3+j\cdot k+k^{3}$ when $j=2$ and $k=6$.
\item[b)] $7-5p+3q$ when $p=1$ and $q=7$.
\item[c)] $10m+\frac{n^{2}}{4}$ when $m=5$ and $n=4$.
\item[d)] $6+\frac{4}{a}+\frac{b}{3}$ when $a=4$ and $b=3$.
\item[e)] $8j-k+14$ when $j=0.25$ and $k=1$.
\item[f)] $6t-20-32u$ when $t=6$ and $u=\frac{1}{4}$.
\item[g)] $\frac{2}{5}g+3h-6$ when $g=10$ and $h=6$.
\item[h)] $\frac{3}{7}r+\frac{5}{8}s$ when $r=14$ and $s=8$.
\end{enumerate}
\end{multicols}
\end{ex}

\begin{sol}
\begin{enumerate}
\item[a)] We proceed as in the examples, and we replace each variable  in $3+j\cdot k+k^{3}$  by its assigned numerical value ($j=2$ and $k=6$):
$$3+j\cdot k+k^{3}=3+2\cdot6+6^{3}=3+12+216=231.$$
\item[b)] Solving the remaining evaluation follows the same argumentation as in a). We have
$$7-5p+3q=7-5\cdot 1+3\cdot7=23.$$
\item[c)] $10m+\frac{n^{2}}{4}=54$.
\item[d)] $6+\frac{4}{a}+\frac{b}{3}=8$.
\item[e)] $8j-k+14=15$.
\item[f)] $6t-20-32u=8$. 
\item[g)] $\frac{2}{5}g+3h-6=16$.
\item[h)] $\frac{3}{7}r+\frac{5}{8}s=11$.
\end{enumerate}
\end{sol}

\section{Linear Equations}
\subsection{Linear Terms}
A \underline{linear term} is an expression of the form $cx$, where $c\in \mathbb{R}$ and $c\neq 0$, is  called the \underline{coefficient}, and $x$ is a variable. 

\textit{Example}: $3x, -5y, z, \frac{1}{2}w, -t$ are linear terms with variables $x,y,z,w$, and $t$, respectively; and coefficients $3, -5, 1, \frac{1}{2}$, and $-1$, respectively.

Linear terms with the same variable can be combined into a single term by addition.

\textit{Example}: 
\begin{equation*}
\begin{split}
& 5x + 4x = (5 + 4)x = 9x.\\
& 3x + 2x - x + 4x = (3 + 2 - 1 + 4)x = 8x.\\
& t + 5t - 8t = (1 + 5 - 8)t = -2t.\\
& -2y +\frac{2}{3}y =(-2 +\frac{2}{3})y= -\frac{4}{3}y.
\end{split}
\end{equation*}
\underline{Note}: We are using the distributive law here. This is possible because, in each expression, the value of the variable is the same wherever it occurs. However, it is not possible to combine linear terms with different variables. 

\textit{Example}: The expressions
$$5x + 3y \text{ or } z - 3t$$
cannot be simplified in any way. This is because $x$ and $y$ (or $z$ and $t$) are independent variables (they don't necessarily take the same value).

Linear terms with the same variable are called \underline{like terms}.

\begin{ex}
Simplify the expressions by combining like terms.
\begin{multicols}{2}
\begin{enumerate}
\item[a)] $2x+9x$.
\item[b)] $-3c + 14c - 19c - d$.
\item[c)] $a - b + 2a - c + 2b$.
\item[d)] $3x +\frac{1}{2}y -\frac{3}{2}x - z -\frac{2}{3}x + 2z$.
\end{enumerate}
\end{multicols}
\end{ex}

\begin{sol}
\begin{enumerate}
\item[a)] We proceed as in the example. We see that the variable $x$ first appears with coefficient $2$ in the linear term $2x$ and then with coefficient $9$ in the linear term $9x$. Therefore, we can combine the two terms and obtain:
$$2x+9x=(2+9)x=11x.$$
\item[b)] This time we have $4$ linear terms in the expression, three with variable $c$: $-3c$, $14c$ and $-19c$; and one with variable $d$: $-d$. Since it is not possible to combine terms with different variables, we can only combine $-3c$, $14c$ and $-19c$ and leave the linear term $-d$ as it is. We obtain:
$$-3c + 14c - 19c - d=(-3c+14c-19c)-d=(-3+14-19)c-d=-8c-d.$$
\item[c)] In this expression, we have $5$ linear terms and $3$ variables. We know we can only combine the terms with the same variable: $a$ and $2a$; and $-b$ and $2b$. We obtain:
 $$a - b + 2a - c + 2b=(a+2a)+(-b+2b)-c=3a+b-c.$$
\item[d)] We have:
$$3x +\frac{1}{2}y -\frac{3}{2}x - z -\frac{2}{3}x + 2z=(3-\frac{3}{2}-\frac{2}{3})x+\frac{1}{2}y+(-1+2)z=\frac{5}{6}x+\frac{1}{2}y + z.$$
\end{enumerate}
\end{sol}

\subsection{Linear Equations in One Variable}
An \underline{equation} is a statement that two mathematical expressions are equal. If the statement has the following form
$$ax+b=0, \text{ where } x \text{ is a variable and } a, b\in \mathbb{R} \text{ with }a\neq 0,$$
then we have a \underline{linear equation in one variable}. 

\textit{Examples}:
\begin{equation*}
\begin{split}
& 2x +4=8.\\
& -3y = 12.\\
&  z - 9 = -1.\\
& 2-\frac{2}{3}t=0.
\end{split}
\end{equation*}
\underline{Note}: a linear equation contains only linear terms, constants and the equality symbol.

A \underline{solution} to an equation in one variable is a number which, when substituted for the variable,
makes a true statement.

\textit{Example}: $2$ is a solution of $2x +4=8$, as $2\cdot 2+4=8$.

\subsection{Solving Linear Equations}
We find solutions to equations using two common-sense principles:
\begin{itemize}
\item Adding equal values to quantities that are equal produces equal quantities:\\
\textit{Example}: To the equality $3=3$ we can add $\textcolor{green}{2}$ on both sides of the equal sign and the equality still keeps: $3+\textcolor{green}{2}=3+\textcolor{green}{2}$.
\item Multiplying quantities that are equal by equal values produces equal quantities:\\
\textit{Example}: We can multiply the equality $3=3$ by $\textcolor{green}{2}$ on both sides of the equal sign and the equality still keeps: $3\cdot \textcolor{green}{2}=3\cdot \textcolor{green}{2}$.
\end{itemize} 
Consider the linear equation
$$x - 3 = 6.$$
Although we cannot say that $x - 3$ and $6$ are “equals” (without knowing the value of $x$) we can say
that if they are equal for some $x$, then adding equals to both sides, or multiplying both sides by equals,
will produce a new pair of equals for the same $x$. In particular, if $x - 3 = 6$ is true for some $x$, so is the equation obtained by adding $3$ to both sides:
$$x-3=6 \textcolor{blue}{\ |+3}$$
$$x-3\textcolor{blue}{+3}=6\textcolor{blue}{+3}$$
$$x=9.$$
The solution to the last equation is obvious (and unique): $9$. It is slightly less obvious that $9$ is a solution of the original equation.  Could there be some other solution to the original equation? Say $p$ is another solution. Then $p-3 = 6$, and adding $3$ to both sides yields $p = 9$. So $p$ is not different from the solution we already found. We conclude that $9$ is the unique solution of $x - 3 = 6$.\\
\underline{Note}: When we use the notation `$| (mathematical \ expression)$' at the end of an equation (example $x-3=6 \ |+3$) means that we will be applying the mathematical expression to both sides of the equal sign. This mathematical expression is usually addition of a term (can be a constant or a linear term) or multiplication by a term (again a constant or a linear term). 

\begin{theorem}
The solution of a linear equation in one variable is unique. 
\end{theorem}

\begin{proof}
Let $ax+b=0$, where $a,b\in \mathbb{R}$ and $a\neq 0$, be a linear equation in one variable. Assume there exist $s_{1}, s_{2} \in \mathbb{R}$ such that both $s_{1}$ and $s_{2}$ are solutions of $ax+b=0$ and $s_{1}\neq s_{2}$. Now, as $s_{1}$ is a solution, we have:
$$a\cdot s_{1}+b=0.$$
Similarly, as $s_{2}$ is a solution, we have:
$$a\cdot s_{2}+b=0.$$
Therefore: $$a\cdot s_{1}+b=a\cdot s_{2}+b.$$
We first add $-b$ to both sides of the equation and then multiply by $\frac{1}{a}$. We obtain: 
\begin{equation*}
\begin{split}
& a\cdot s_{1}+b=a\cdot s_{2}+b \textcolor{blue}{\ | -b}\\
& a\cdot s_{1}+b\textcolor{blue}{-b}=a\cdot s_{2}+b\textcolor{blue}{-b}\\
& a\cdot s_{1}=a\cdot s_{2} \textcolor{blue}{\ | \cdot \frac{1}{a}}\\
& a\textcolor{blue}{\cdot \frac{1}{a}}\cdot s_{1}=a\textcolor{blue}{\cdot \frac{1}{a}}\cdot s_{2}\\
& s_{1}=s_{2}.
\end{split}
\end{equation*}
which contradicts our assumption. \\
In conclusion, we have shown that the solution of a linear equation in one variable is unique.
\end{proof}

In the following table we outline the algorithm that determines the solution of a linear equation with one variable given in the standard form: $ax+b=0$, where $a, b\in \mathbb{R}$ and $a\neq 0$.\\
\underline{Recall}: The solution of a linear equation with one variable is unique.\\
On the right we see the algorithm explained for the example: $\frac{1}{7}x-\frac{3}{2}=0$.

\begin{table*}[h!]
\centering
\begin{tabular}{ l | l }
Algorithm & Example \\
\hline
\multirow{5}{16em}{Add $-b$ to both sides of the equation $ax+b=0$.}. & In our example $a=\frac{1}{7}$ and $b=-\frac{3}{2}$.\\
	 & Add $-(-\frac{3}{2})=\frac{3}{2}$ to both sides of the equation $\frac{1}{7}x-\frac{3}{2}=0$:\\
      & $\frac{1}{7}x-\frac{3}{2}=0 \textcolor{blue}{\ | + \frac{3}{2}}$\\
      & $\frac{1}{7}x-\frac{3}{2} \textcolor{blue}{+ \frac{3}{2}}=0 \textcolor{blue}{+ \frac{3}{2}}$\\
      & $\frac{1}{7}x=\frac{3}{2}$.
\\\hline
\multirow{4}{16em}{Multiply both sides of the equation $ax=-b$ by $\frac{1}{a}$.} & Multiply both sides of the equation $\frac{1}{7}x=\frac{3}{2}$ by $\frac{1}{\frac{1}{7}}=7$:\\
      & $\frac{1}{7}x=\frac{3}{2} \textcolor{blue}{\ | \cdot 7}$\\
      & $\frac{1}{7} \textcolor{blue}{\cdot 7}\cdot x=\frac{3}{2} \textcolor{blue}{\cdot 7}$\\
      & $x=\frac{21}{2}$. \\\hline
\multirow{2}{16em}{Obtain the solution $-\frac{b}{a}$.}& We verify that  $\frac{21}{2}$ is indeed a solution of $\frac{1}{7}x-\frac{3}{2}=0$:\\
       & $\frac{1}{7}\cdot \textcolor{blue}{\cdot \frac{21}{2}}-\frac{3}{2}=\frac{3}{2}-\frac{3}{2}=0$.
\\\hline
\end{tabular}
\end{table*}

\newpage
\begin{ex}
Verify that $-\frac{b}{a}$ is the solution of $ax+b=0$, where $a, b\in \mathbb{R}$ and $a\neq 0$.
\end{ex}

\begin{sol}
To verify that $-\frac{b}{a}$ is the solution of $ax+b=0$, we need to evaluate the expression $ax+b$ for $x=-\frac{b}{a}$ and check to see if the result is $0$. We have:
$$ax+b= a\cdot(-\frac{b}{a})+b=-b+b=0.$$
We conclude that $-\frac{b}{a}$ is the solution of $ax+b=0$.
\end{sol}

\begin{ex}
Find the solution of the following equations.
\begin{enumerate}
\item[a)] $9=-\frac{1}{5}z$.
\item[b)] $3y-3=6$.
\item[c)] $\frac{1}{2}x-2=\frac{3}{4}$.
\item[d)] $7t-13=1$.
\end{enumerate}
\end{ex}

\begin{sol}
\begin{enumerate}
\item[a)] We see that in order to solve the equation $9=-\frac{1}{5}z$ we only need to multiply both sides of the equation by $-5$. We have:
\begin{equation*}
\begin{split}
& 9=-\frac{1}{5}z \textcolor{blue}{\ | \cdot (-5)}\\
& 9\textcolor{blue}{\cdot (-5)}=-\frac{1}{5}\textcolor{blue}{\cdot (-5)}\cdot z\\
& -45=z
\end{split}
\end{equation*}
The solution of the equation $9=-\frac{1}{5}z$ is $45$.
\item[b)] To solve $3y-3=6$ we first need to add $3$ to both sides of the equation and then multiply by $\frac{1}{3}$:
\begin{equation*}
\begin{split}
& 3y-3=6 \textcolor{blue}{\ | +3}\\
& 3y-3\textcolor{blue}{+3}=6\textcolor{blue}{+3}\\
& 3y=9\textcolor{blue}{\ | \cdot\frac{1}{3}}\\
& 3\textcolor{blue}{\cdot\frac{1}{3}}\cdot y=9\textcolor{blue}{\cdot\frac{1}{3}}\\
& y=3
\end{split}
\end{equation*}
The solution of the equation $3y-3=6$ is $3$.
\item[c)] We proceed as in Item b). We have:
\begin{equation*}
\begin{split}
& \frac{1}{2}x-2=\frac{3}{4} \textcolor{blue}{\ | +2}\\
& \frac{1}{2}x=\frac{11}{4}\textcolor{blue}{\ | \cdot 2}\\
& x=\frac{11}{2}
\end{split}
\end{equation*}
The solution of the equation $\frac{1}{2}x-2=\frac{3}{4}$ is $\frac{11}{2}$.
\item[(d)] We proceed as in Item b). We have:
\begin{equation*}
\begin{split}
&7t-13=1 \textcolor{blue}{\ | +13}\\
& 7t=14\textcolor{blue}{\ | \cdot \frac{1}{7}}\\
& t=2
\end{split}
\end{equation*}
The solution of the equation $7t-13=1$ is $2$.
\end{enumerate}
\end{sol}

In the following table we outline the algorithm that determines the solution of a linear equation with one variable given in any form. The strategy is to systematically transform the linear equation into simpler equivalent equations, until we arrive at an equation of the form $x=c$, where $c\in \mathbb{R}$, whose solution is obvious. \\
\underline{Recall}: The solution of a linear equation with one variable is unique.\\
\underline{Recall}: Linear terms with the same variable can be combined into a single term by addition.\\
On the right we see the algorithm explained for the example: $x-\frac{1}{2}x+21-3x+17+5(x-6)=2x+\frac{1}{5}-\frac{1}{2}x+17+2(x-\frac{1}{3})$.

\begin{table*}[h!]
\centering
\begin{tabular}{ l | l }
Algorithm & Example \\
\hline
\multirow{6}{15em}{(When possible) Simplify the expression.} & The term $-\frac{1}{2}x+17$ appears on both sides of the equal sign:\\
		& $x\textcolor{blue}{-\frac{1}{2}x}+21-3x\textcolor{blue}{+17}+5(x-6)=2x+\frac{1}{5}\textcolor{blue}{-\frac{1}{2}x+17}+2(x-\frac{1}{3})$.\\
		& Before we do anything else, we remove this term\\
		& from the expression in order to simplify (we are simply\\
		& adding $-(-\frac{1}{2}x+17)$ to both sides of the equation).\\
		& We obtain: $x+21-3x+5(x-6)=2x+\frac{1}{5}+2(x-\frac{1}{3})$.\\\hline
\multirow{4}{15em}{Distribute when needed (open the parentheses by doing the multiplication).}& On the left side of the equal sign we open the bracket $5(x-6)$:\\
       & $x+21-3x+5(x-6)=x+21-3x+5x-30$.\\
       & On the right side of the equal sign we open the bracket $2(x-\frac{1}{3})$:\\
       & $2x+\frac{1}{5}+2(x-\frac{1}{3})=2x+\frac{1}{5}+2x-\frac{2}{3}$.
\\\hline
\multirow{4}{15em}{Combine the terms with the same variable into a single term on each side of the equal sign.} & On the left side of the equal sign we have:\\
		& $x+21-3x+5x-30=(x-3x+5x)+(21-30)=3x-9$.\\
		& On the right side of the equal sign we have:\\
		& $2x+\frac{1}{5}+2x-\frac{2}{3}=(2x+2x)+(\frac{1}{5}-\frac{2}{3})=4x-\frac{7}{15}$.
		\\\hline
\multirow{6}{15em}{Add the additive inverse of terms to both sides in order to bring your equation to the standard form $ax+b=0$, where $a,b\in \mathbb{R}$ and $a\neq 0$.}& We have arrived at the equation:\\
		& $3x-9=4x-\frac{7}{15}$.\\
		& We add $-(3x-9)=-3x+9$ to both sides of the equation in\\
		& order to write the equation in standard form:\\
		&  $3x-9=4x-\frac{7}{15}\textcolor{blue}{\ | -3x+9}$.\\
		& $3x\textcolor{blue}{-3x}-9\textcolor{blue}{+9}=4x\textcolor{blue}{-3x}-\frac{7}{15}\textcolor{blue}{+9}$.\\
		& $0=x+\frac{128}{15}$.
\\\hline
\multirow{2}{15em}{The solution of $ax+b=0$ is $-\frac{b}{a}$.}& We add $-\frac{128}{15}$ to both sides of $0=x+\frac{128}{15}$ and we obtain\\
		& the solution $-\frac{128}{15}$.
\\\hline
\end{tabular}
\end{table*}

\newpage
\textit{Example $1$}: To solve the equation $ 2(x + 3) = -3x + 10+x-6$, we 
\begin{itemize}
\item combine the terms with the same variable and get $2(x+3)=-2x+4$.
\item distribute $2(x+3)$ and get: $2x+6=-2x+4$.
\item add the additive inverse of terms in order to bring $2x+6=-2x+4$ to the form $ax+b=0$, where $a,b\in \mathbb{R}$ and $a\neq 0$. 
\begin{equation*}
\begin{split}
& 2x+6=-2x+4 \textcolor{blue}{\ | +2x-4}\\
& 2x\textcolor{blue}{+2x}+6\textcolor{blue}{-4}=-2x\textcolor{blue}{+2x}+4\textcolor{blue}{-4}\\
& 4x+2=0\\
\end{split}
\end{equation*}
\item Since that the solution of $ax+b=0$ is $-\frac{b}{a}$, the solution for our equation $4x+2=0$ ($a=4$ and $b=2$) is $x=-\frac{2}{4}=-\frac{1}{2}$.
\end{itemize}

\textit{Example $2$}: To solve the equation $5y-10=3y+4+4(y-2)$, we
\begin{itemize}
\item simplify the equation: we see that on the left side of the equation we have $5y-10=5(y-2)$ and that on the right side we have the term $4(y-2)$. Thus, we can add $-4(y-2)$ to both sides of the equation to simplify the expression:
\begin{equation*}
\begin{split}
& 5y-10=3y+4+4(y-2) \textcolor{blue}{\ | -4(y-2)}\\
& 5(y-2) \textcolor{blue}{\ -4(y-2)}=3y+4\\
& y-2=3y+4\\
\end{split}
\end{equation*}
\item add the additive inverse of terms in order to bring $y-2=3y+4$ to the form $ay+b=0$, where $a,b\in \mathbb{R}$ and $a\neq 0$.
\begin{equation*}
\begin{split}
& y-2=3y+4 \textcolor{blue}{\ | -y+2}\\
& y\textcolor{blue}{-y}-2\textcolor{blue}{+2}=3y\textcolor{blue}{-y}+4\textcolor{blue}{+2}\\
& 0=2y+6.
\end{split}
\end{equation*}
\item Since that the solution of $ay+b=0$ is $-\frac{b}{a}$, the solution for our equation $2y+6=0$ ($a=2$ and $b=6$) is $y=-\frac{6}{2}=-3$.
\end{itemize}

\textit{Example $3$}: Solve $3(6-z)-4=3z-4$.
\begin{itemize}
\item We simplify the equation: both sides of the equation have the form $3*\triangle -4=3*\square -4$. Instead of inspecting $\triangle$ and $\square$, we simplify. First we add $4$ to both sides of the equation and then we multiply both sides of the equation by $\frac{1}{3}$:
\begin{equation*}
\begin{split}
& 3(6-z)-4=3z-4 \textcolor{blue}{\ | +4}\\
& 3(6-z)=3z\textcolor{blue}{\ | \cdot \frac{1}{3}}\\\
& 6-z=z\\
\end{split}
\end{equation*}
\item We now bring $6-z=z$ to the form $az+b=0$, where $a,b\in \mathbb{R}$ and $a\neq 0$.
\begin{equation*}
\begin{split}
& 6-z=z \textcolor{blue}{\ |-z}\\
& 6-z\textcolor{blue}{-z}=z\textcolor{blue}{-z}\\
& 6-2z=0.
\end{split}
\end{equation*}
\item Since that the solution of $az+b=0$ is $-\frac{b}{a}$, the solution for our equation $6-2z=0$ ($a=-2$ and $b=6$) is $z=-\frac{6}{-2}=3$.
\end{itemize}

\begin{ex}
Find the solution of the following equations.
\begin{enumerate}
\item[a)] $9=-\frac{1}{5}z+2z-11$.
\item[b)] $3y-3=6-9y$.
\item[c)] $\frac{1}{2}x-2+2x=\frac{3}{4}-x$.
\item[d)] $7t-13=1-2t+11t-6$.
\item[e)] $9w-7=7w-11$.
\item[f)] $20-6+4x=2-2x$.
\end{enumerate}
\end{ex}

\begin{sol}
\item[a)] We combine the linear terms and we add $-9$ to both sides of the equation:
\begin{equation*}
\begin{split}
& 9=(-\frac{1}{5}+2)z-11\textcolor{blue}{\ | -9}\\
& 0=\frac{9}{5}z-20
\end{split}
\end{equation*}
The solution of $\frac{9}{5}z-20=0$ is $-\frac{-20}{\frac{9}{5}}=\frac{20\cdot 5}{9}=\frac{100}{9}$ (look at the algorithm to solve linear equation with one variable given in the standard form $ax+b=0$, $a,b\in \mathbb{R}$ with $a\neq 0$).
\item[b)] We add $9y-6$ to both sides of the equation:
\begin{equation*}
\begin{split}
& 3y-3=6-9y\textcolor{blue}{\ | +9y-6}\\
& 3y\textcolor{blue}{+9y}-3\textcolor{blue}{-6}=6\textcolor{blue}{-6}-9y\textcolor{blue}{+9y}\\
& 12y-9=0.
\end{split}
\end{equation*}
The solution of $12y-9=0$ is $-\frac{-9}{12}=\frac{9}{12}=\frac{3}{4}$ (look at the algorithm to solve linear equation with one variable given in the standard form $ax+b=0$, $a,b\in \mathbb{R}$ with $a\neq 0$).
\item[c)] From this point onward, we will stop explicitly writing the operations we are making. They should be clear at each step. Furthermore, once we bring the equation at the standard form, the solution should be obvious. You might find it useful to explain what is happening next to each transformation of the equation.
\begin{equation*}
\begin{split}
& \frac{1}{2}x-2+2x=\frac{3}{4}-x\\
& \frac{5}{2}x-2=\frac{3}{4}-x\\
& \frac{5}{2}x-2+x-\frac{3}{4}=0\\
& \frac{7}{2}x-\frac{11}{4}=0\\
& x= \frac{11}{14}.
\end{split}
\end{equation*}
\item[d)] \begin{equation*}
\begin{split}
& 7t-13=1-2t+11t-6\\
& 7t-13=9t-5\\
& 0=2t+8\\
& t=-4.
\end{split}
\end{equation*}
\item[e)] \begin{equation*}
\begin{split}
& 9w-7=7w-11\\
& 2w+4=0\\
& w=-2.
\end{split}
\end{equation*}
\item[f)] \begin{equation*}
\begin{split}
& 20-6+4x=2-2x\\
& 14+4x=2-2x\\
& 6x+12=0\\
& x=-2.
\end{split}
\end{equation*}
\end{sol}

\section{Linear Inequalities}
In this section we look at inequality and the symbols $<$, $>$, $\leq$ and $\geq$. 

Let's start with the inequality statement $x>5$. What value could $x$ take in order to make the inequality $x>5$ true? One option is $x=6$. But other options could be $6, 31, 1000000$, or even $5.001$. The number of solutions is \underline{infinite}: any number greater than $5$ is a solution to the inequality $x>5$. 

We can represent inequalities using interval notation.  In interval notation, we express $x>5$ as $x\in (5, \infty)$. The symbol $\infty$ represents "infinity." Infinity is not an actual number. 

The meaning of each bracket in interval notation is given in the table below:

\begin{table*}[h!]
\centering
\begin{tabular}{ c | l  | l}
Symbol & Meaning & Example \\
\hline
\multirow{3}{1.5em}{$(a, b)$} & the set of real numbers  that are  strictly greater & $(2,3)$: the set of real numbers $c$ that\\
											 & than $a$ and strictly less than $b$  & are strictly greater than $2$: $c>2$; and \\
											 & & strictly less than $3$: $c<3$.\\\hline
\multirow{3}{1.5em}{$[a, b]$} & the set of real numbers that are greater or equal & $[2,3]$: the set of real numbers $c$ that\\
											 & to $a$ and less or equal to $b$  & are greater or equal to $2$: $c\geq 2$; and \\
											 & & less or equal to $3$: $c\leq 3$.\\\hline
\multirow{3}{1.5em}{$(a,b]$} & the set of real numbers  that are strictly greater & $(2,3]$: the set of real numbers $c$ that\\
											 & than $a$ and less or equal to $b$  & are strictly greater than $2$: $c>2$; and\\
											 & & less or equal to $3$: $c\leq 3$.\\\hline
\multirow{3}{1.5em}{$[a,b)$} & the set of real numbers  that are greater or equal & $[2,3)$: the set of real numbers $c$ that\\
											 & to $a$ and strictly less than $b$  & are greater or equal to $2$: $c\geq 2$; and\\
											 & & strictly less than $3$: $c<3$.\\\hline
\end{tabular}
\end{table*}

\begin{ex}
\begin{enumerate}
\item[a)] Find the numbers that are in $[2,3]$ but not in $(2,3)$.
\item[b)] Find the numbers that are in $(2,3)$ but not in $[2,3]$.
\item[c)] Find the numbers that are in $(2,3]$ but not in $[2,3)$.
\item[d)] Find the numbers that are in $[2,3)$ but not in $(2,3]$.
\item[e)] Find the numbers that are in $[2,3]$ but not in $[2,3)$.
\item[f)] Find the numbers that are in $(2,3)$ but not in $(2,3]$.
\end{enumerate}
\end{ex}

\begin{sol}
\begin{enumerate}
\item[a)]The set $[2,3]$ contains all the real numbers $c$ with the property that $2\leq c\leq 3$. On the other hand, the set $(2,3)$ contains all the real numbers $c$ with the property that $2< c< 3$. Thus, the numbers $c$ that are in $[2,3]$ but not in $(2,3)$ are $2$ and $3$.
\item[b)] We have seen that the set $(2,3)$ contains all the real numbers $c$ with the property that $2< c< 3$, while the set $[2,3]$ contains all the real numbers $c$ with the property that $2\leq c\leq 3$. Therefore there are no numbers $c$ in $(2,3)$ that are not in $[2,3]$.
\item[c)] The set $(2,3]$ contains all the real numbers $c$ with the property that $2< c\leq 3$. On the other hand, the set $[2,3)$ contains all the real numbers $c$ with the property that $2\leq c< 3$. Thus, $3$ is the only number that is in $(2,3]$ but not in $[2,3)$.
\item[d)] We argue as in Item c) to show that $2$ is the only number that is in $[2,3)$ and not in $(2,3]$.
\item[e)] Using the description of $[2,3]$ from Item a) and that of $[2,3)$ from Item c), we see that $3$ is the only number that is in $[2,3]$ and not in $[2,3)$. 
\item[f)] Using the description of $(2,3)$ from Item a) and that of $(2,3]$ from Item c), we see that there are no numbers $c$ in $(2,3)$ that are not in $(2,3]$. 
\end{enumerate}
\end{sol}

\begin{ex} Represent the following inequalities using the interval notation:
\begin{multicols}{3}
\begin{enumerate}
\item[a)] $x>3$.
\item[b)] $x<-2$.
\item[c)] $-2<x<3$.
\item[d)]$x\geq -6$.
\item[e)] $2\leq x$.
\item[f)] $2\leq x\leq 6$.
\item[g)] $4\geq x>-2$.
\item[h)] $-8<x\leq 9$.
\item[i)] $0<x<1$.
\end{enumerate}
\end{multicols}
\end{ex}

\begin{sol}
\begin{multicols}{2}
\begin{enumerate}
\item[a)] $x\in (3,\infty)$.
\item[b)] $x\in (-\infty,-2)$.
\item[c)] $x\in (-2,3)$.
\item[d)]$x\in [-6,\infty)$.
\item[e)] $x\in [2,\infty)$.
\item[f)] $x\in [2,6]$.
\item[g)] $x\in (-2,4]$.
\item[h)] $x\in (-8,9]$.
\item[i)] $x\in (0,1)$.
\end{enumerate}
\end{multicols}
\end{sol}

\subsection{Solving Linear Inequalities}
A linear inequality is much like a linear equation, where the equal sign is replaced with an inequality sign. A \underline{linear inequality in one variable} is an inequality written in one of the forms:
$$ ax+b<c, \ ax+b\leq c, \ ax+b>c, \text{ or }ax+b\geq c $$
where $a, b, c\in \mathbb{R}$ and $a\neq 0$.

A \underline{solution} to a linear inequality is a number that when substituted for the variable produces a true statement. Linear inequalities have either infinitely many solutions or no solution. If there are infinitely many solutions, we express this using interval notation.

When solving linear equations, we used the properties of equality to add and multiply by the same number both sides of the equation and still keep the equality. Similar properties hold true for inequalities. We use these properties to obtain a simplified equivalent inequality, with the same solution set.

\begin{itemize}
\item We can \underline{add the same number} to both sides of the inequality and the inequality remains true.
\end{itemize}

\textit{Example:} We know that $2<3$ is true. If we add $5$ to both sides of the inequality, the statement remains true:
$$2<3 \textcolor{blue}{\ |+5}$$
$$7<8.$$

\begin{formula}
Let $a,b, c\in \mathbb{R}$. Then:
\begin{enumerate}
\item[(i)] If $a<b$, we have $a+c<b+c$.
\item[(ii)] If $a>b$, we have $a+c>b+c$.
\end{enumerate}
\end{formula}

\begin{itemize}
\item We can \underline{multiply by a \textbf{positive} number} both sides of the inequality and the inequality remains true.
\end{itemize}

\textit{Example:} If we multiply both sides of the inequality $2<3$ by $5$, the statement remains true:
$$2<3 \textcolor{blue}{\ |\cdot 5}$$
$$10<15.$$

\begin{formula}
Let $a,b, c\in \mathbb{R}$ with $c>0$. Then
\begin{enumerate}
\item[(i)] If $a<b$, we have $a\cdot c<b\cdot c$.
\item[(ii)] If $a>b$, we have $a\cdot c>b\cdot c$.
\end{enumerate}
\end{formula}

\begin{itemize}
\item We can \underline{multiply by a \textbf{negative} number} both sides of the inequality and the \textbf{inequality sign reverses}.
\end{itemize}

\textit{Example:} If we multiply both sides of the inequality $2<3$ by $-5$, the inequality sign reverses:
$$2<3 \textcolor{blue}{\cdot \ | \cdot(-5)}$$
$$-10>-15.$$

\begin{formula}
Let $a,b, c\in \mathbb{R}$ with $c<0$. Then
\begin{enumerate}
\item[(i)] If $a<b$, we have $a\cdot c>b\cdot c $.
\item[(ii)] If $a>b$, we have $a\cdot c<b\cdot c$.
\end{enumerate}
\end{formula}

\textit{Example $1$}: We want to solve the linear inequality $2x+6\leq 4$.
\begin{itemize}
\item We add $-6$ to both sides of the inequality (the inequality sign stays the same):
$$2x+6\leq 4 \textcolor{blue}{\ |-6}$$
$$2x\leq -2.$$
\item We multiply both sides of the inequality by $\frac{1}{2}$ (the inequality sign stays the same as $\frac{1}{2}>0$):
$$2x\leq -2 \textcolor{blue}{\ |\cdot \frac{1}{2}}$$
$$x\leq -1.$$
\item We have obtained the solution $x\in (-\infty, -1]$.
\end{itemize}


\textit{Example $2$}: Solve the linear inequality $-2x+6\leq 4$.
\begin{itemize}
\item As before, we add $-6$ to both sides of the inequality:
$$-2x+6\leq 4 \textcolor{blue}{\ |-6}$$
$$-2x\leq -2.$$
\item We multiply both sides of the inequality by $-\frac{1}{2}$ and we note that the inequality sign \textbf{reverses} as $-\frac{1}{2}<0$:
$$-2x\leq -2 \textcolor{blue}{\ |\cdot (-\frac{1}{2})}$$
$$x\geq 1.$$
\item We have obtained the solution $x\in [1, \infty)$.
\end{itemize}

A \underline{compound inequality} is two inequalities in one statement joined by the word 'and' or by the word 'or'. 

\textit{Example of a compound inequality with the word 'and'}: The compound inequality 
$$-13<3x-7<17$$
expresses that the two inequalities $-13<3x-7$ and $3x-7<17$ hold at the same time. The solution set will be the \underline{intersection of the two individual solution sets} (the intersection of the solution set of $-13<3x-7$ and the solution set of $3x-7<17$). This is because we want the statement to be true for both inequalities. Thus, from the solution set of $-13<3x-7$ we have to remove the ones that are not solutions of $3x-7<17$, and vice versa. \\
\underline{Algorithm}:
\begin{itemize}
\item We first solve $-13<3x-7$ as in \textit{Example $1$.}, and obtain $x>-2$. Therefore, the solution set expressed in interval notation is $(-2, \infty)$.\\
\underline{Remark}: We see that $x=-1$ belongs to the solution set $(-2, \infty)$, hence is a solution of $-13<3x-7$. Moreover, as $3\cdot (-1)-7=-8<17$, it also belongs to the solution set of $3x-7<17$. However, $x=9$ is a solution of $-13<3x-7$, as it belongs to the solution set $(-2, \infty)$, but it is not a solution of $3x-7<17$, as $3\cdot 9-7=20>17$. This underlines the fact that from the solution set of $-13<3x-7$ we need to remove the ones that are not solutions of $3x-7<17$.
\item We solve $3x-7<17$, and obtain $x<8$. Therefore, the solution set expressed in interval notation is $(-\infty, 8)$.\\
\underline{Remark}: We have seen that $x=-1$ belongs to both the solution of $-13<3x-7$ and to the solution set of $3x-7<17$. However, $x=-3$ is a solution of $3x-7<17$, as it belongs to the solution set $ (-\infty, 8)$, but it is not a solution of $-13<3x-7$, as $3\cdot (-3)-7=-16<-13$. This underlines the fact that from the solution set of $3x-7<17$ we need to remove the ones that are not solutions of $-13<3x-7$.
\item With the two remarks in mind, we conclude that the solution set of the compound inequality $-13<3x-7<17$ is the intersection of the two individual solution sets $(-2, \infty)$ and $(-\infty, 8)$: 
$$x\in (-2, \infty) \cap  (-\infty, 8)=(-2, 8).$$
\end{itemize}

\underline{Remark}: In \textit{Example $3$} you will see how to solve compound inequalities with the word 'and' at the same time (without treating the two inequalities separately).

\textit{Example of a compound inequality with the word 'or'}: Solve $3x-7\geq -2$ or $-2x+3\geq 5$.
\begin{itemize}
\item For compound inequalities with the word 'or' you treat both inequalities separately and the solution set will be the union of the individual solution sets. This is because we want at least one of the inequalities to hold true. Thus we need to consider all the values of $x$ for which either $3x-7\geq -2$ or $-2x+3\geq 5$ holds (or both at the same time). 
\item We solve the inequality $3x-7\geq -2$. We proceed as in \textit{Example $1$}, and obtain that $x\geq \frac{5}{3}$, therefore the solution set is $[\frac{5}{3}, \infty)$.\\
\underline{Remark}: Any $x\in [\frac{5}{3}, \infty)$ is a solution of the compound inequality $3x-7\geq -2$ or $-2x+3\geq 5$, as it is a solution of $3x-7\geq -2$. However, this is not the complete set of solutions of $3x-7\geq -2$ or $-2x+3\geq 5$, as there might still be solutions of $-2x+3\geq 5$ that are not solutions of $3x-7\geq -2$.
\item We solve $-2x+3\geq 5$, as in \textit{Example $2$}, and obtain that $x\leq -1$. Thus the solution set is $(-\infty, -1]$. 
\item Since we have determined the individual solutions sets of both $3x-7\geq -2$ and $-2x+3\geq 5$, we can conclude that the solution set of the compound inequality $3x-7\geq -2$ or $-2x+3\geq 5$ is $x\in  (-\infty, -1] \cup [\frac{5}{3}, \infty)$.
\end{itemize}

\textit{Example $3$}: Solve the linear inequality $2\leq -3x+7<20$.
\begin{itemize}
\item In the expression $2\leq -3x+7<20$ we notice that we have two inequality signs. We have seen previously that this means that the two inequalities $2\leq -3x+7$ and $-3x+7<20$ have to hold at the same time. When treating individual inequalities, we know that we can add to or multiply by terms both sides of the inequality and the inequality keeps or reverses its sign. We can do the same when we have multiple inequality signs and the rules remain the same:
\begin{itemize}
\item addition of a term keeps the inequality sign;
\item multiplication by a positive term keeps the inequality sign;
\item multiplication by a negative term reverses the inequality sign.
\end{itemize}
\item When solving either $2\leq -3x+7$ or $-3x+7<20$, the first step is to add $7$ to both sides of the respective inequality. The resulting inequalities are $-5\leq -3x$ and $-3x<13$, and they have to hold at the same time. We can express this through the compound inequality$-5\leq -3x<13$. Thus, we see that when applying addition (the same holds true for multiplication) to a compound inequality, we obtain another compound inequality. Thus, instead of performing these operations on the individual inequalities, we can do it all in one go.\\
\underline{Attention}: The operation has to performed \textbf{once} on \textbf{all} sides of the compound inequality: left side, middle side and right side. \\
Coming back to our case, we want to add $7$ on all sides of our compound inequality:
\begin{equation*}
\begin{split}
& 2\leq -3x+7<20 \textcolor{blue}{\ | -7}\\
& 2\textcolor{blue}{-7}\leq -3x+7\textcolor{blue}{-7}<20\textcolor{blue}{-7}\\
& -5\leq -3x\leq 13.
\end{split}
\end{equation*}
We remark that the compound inequality we have obtained consists of the exact two inequalities we get when performing the addition separately.
\item We now want to multiply the inequality $ -5\leq -3x\leq 13$ by $-\frac{1}{3}$. This means that we multiply both sides of $ -5\leq -3x$ by $-\frac{1}{3}$ and both sides of $-3x\leq 13$ by $-\frac{1}{3}$. When working directly on the compound inequality, we have to make sure to effectuate the multiplication on \textbf{all} sides of the inequality: left, middle and right. We also have to make sure to reverse \textbf{all} inequality signs as we are multiplying by $-\frac{1}{3}<0$:
\begin{equation*}
\begin{split}
& 5\leq -3x<13 \textcolor{blue}{\ |\cdot (-\frac{1}{3})}\\
& 5 \textcolor{blue}{\cdot (-\frac{1}{3})}\textcolor{red}{\geq} -3 \textcolor{blue}{\cdot (-\frac{1}{3})}\cdot x\textcolor{red}{>}13 \textcolor{blue}{\cdot (-\frac{1}{3})}\\
&-\frac{5}{3}\textcolor{red}{\geq} x\textcolor{red}{>}-\frac{13}{3}.
\end{split}
\end{equation*}
\item We have obtained the solution set $(-\frac{13}{3}, -\frac{5}{3}]$. 
\end{itemize}

\begin{ex}
Solve the following linear equations.
\begin{multicols}{2}
\begin{enumerate}
\item[a)] $9x\geq 72$.
\item[b)] $y+7\leq -2$.
\item[c)] $7<2z+3\leq 11$.
\item[d)] $-3x+2\geq -1$.
\item[e)] $-2\leq -2y-2\leq 0$.
\item[f)]  $4x+15\leq -1$ or $3x-8\geq -11$.
\end{enumerate}
\end{multicols}
\end{ex}

\begin{sol}
\begin{enumerate}
\item[a)] To solve $9x\geq 72$ we need to multiply both sides of the inequality by $\frac{1}{9}$. We note that the inequality sign stays the same as $\frac{1}{9}>0$:
\begin{equation*}
\begin{split}
& 9x\geq 72 \textcolor{blue}{\ |\cdot (\frac{1}{9})}\\
& 9\textcolor{blue}{\cdot (\frac{1}{9})}\cdot x\geq 72\textcolor{blue}{\cdot (\frac{1}{9})}\\
&x\geq 8.
\end{split}
\end{equation*}
We have obtained the solution set $[8, \infty)$.
\item[b)] To solve $y+7\leq -2$ we need to add $-7$ to both sides of the inequality:
\begin{equation*}
\begin{split}
& y+7\leq -2 \textcolor{blue}{\ |-7}\\
& y+7\textcolor{blue}{-7}\leq -2\textcolor{blue}{-7}\\
&y\leq -9.
\end{split}
\end{equation*}
We have obtained the solution set $(-\infty, -9]$.
\item[c)] To solve the compound inequality $7<2z+3\leq 11$ we first to add $-3$ to all sides of the inequality:
\begin{equation*}
\begin{split}
& 7<2z+3\leq 11 \textcolor{blue}{\ |-3}\\
& 7\textcolor{blue}{-3}<2z+3\textcolor{blue}{-3}\leq 11\textcolor{blue}{-3}\\
&4<2z\leq 8.
\end{split}
\end{equation*}
We now have to multiply the compound inequality $4<2z\leq 8$ by $\frac{1}{2}$. We remark that, as $\frac{1}{2}>0$, the inequality signs stay the same:
\begin{equation*}
\begin{split}
& 4<2z\leq 8 \textcolor{blue}{\ |\cdot \frac{1}{2}}\\
& 4\textcolor{blue}{\cdot \frac{1}{2}}\textcolor{blue}{<}2\textcolor{blue}{\cdot \frac{1}{2}}\cdot z\textcolor{blue}{\leq} 8\textcolor{blue}{\cdot \frac{1}{2}}\\
&2<z\leq 4.
\end{split}
\end{equation*}
We have obtained the solution set $(2,4]$.
\item[d)] We have that
\begin{equation*}
\begin{split}
& -3x+2\geq -1 \textcolor{blue}{\ |-2}\\
& -3x\geq -3\textcolor{blue}{\ |\cdot (-\frac{1}{3})}\\
&x\leq 1.
\end{split}
\end{equation*}
We have obtained the solution set $(-\infty,1]$.
\item[e)] We have that
\begin{equation*}
\begin{split}
& -2\leq -2y-2\leq 0 \textcolor{blue}{\ |+2}\\
& 0\leq -2y\leq 2\textcolor{blue}{\ |\cdot (-\frac{1}{2})}\\
&0\geq y\geq -1.
\end{split}
\end{equation*}
We have obtained the solution set $[-1,0]$.
\item[f)] We first solve $4x+15\leq -1$:
\begin{equation*}
\begin{split}
& 4x+15\leq -1 \textcolor{blue}{\ |-15}\\
& 4x\leq -16 \textcolor{blue}{\ |\cdot \frac{1}{4}}\\
&x\leq -4
\end{split}
\end{equation*}
We have obtained the solution set $(-\infty,-4]$. We now solve $3x-8\geq -11$:
\begin{equation*}
\begin{split}
& 3x-8\geq -11 \textcolor{blue}{\ |+8}\\
& 3x\geq -3\textcolor{blue}{\ |\cdot \frac{1}{3}}\\
&x\geq -1
\end{split}
\end{equation*}
We have obtained the solution set $[-1,\infty)$. Thus, the solution set of the compound inequality $4x+15\leq -1$ or $3x-8\geq -11$ is $(-\infty,-4]\cup [-1,\infty)$
\end{enumerate}
\end{sol}

\begin{ex}\label{Ex6} Applications:
\begin{enumerate}
\item[a)] A teacher won a mini grant of $4,000$ chf to buy tablets for their classroom. The total cost for a tablet is $254.12$ chf. What is the maximum number of tablets the teacher can buy?
\item[b)] Alice's phone plan costs her $28.80$ chf per month plus $0.20$ chf per text message. How many text messages can she send in order to keep her monthly phone bill no more than $50$ chf? 
\item[c)] Bob's best friend is having a destination wedding and the event will last $3$ days and $3$ nights. Bob has $500$ chf in savings and can earn $15$ chf/h babysitting. He expects to pay $350$ chf for airfare, $375$ chf for food, and $60$ chf/night for the hotel room. How many hours must he babysit to have enough money for the trip?
\end{enumerate}
\end{ex}

\begin{sol}
\item[a)] Let us denote by $x$ the maximum number of tablets the teacher can buy. The total costs of the tablets will be $254.12\cdot x$ chf. As the teacher needs this cost to be less or equal to $4,000$ chf, the teacher arrives at the following inequality:
$$254.12\cdot x\leq 4000$$
with solution $x\leq \frac{4000}{254.12}\approx 15.74$. As you can only buy integer values, the maximum number of tables the teacher can buy is $15$.
\item[b)] Every month, Alice has to pay her phone plan of $28.80$ chf (which is a constant cost) and the cost for the text messages she uses which is $0.20\cdot y$, where $y$ is the number of text messages she uses that month. Thus, monthly, Alice has to pay $28.80+ 0.20\cdot y$ chf. As her budget is $50$ chf, that means that Alice has to ensure that the value of $28.80+ 0.20\cdot y$ is less or equal to $50$. To figure out how many text messages she can use, Alice has to solve the inequality:
\begin{equation*}
\begin{split}
& 28.80+ 0.20\cdot y\leq 50 \\
& 0.20\cdot y\leq 21.20\\
&y\leq 106
\end{split}
\end{equation*}
Thus, Alice can use a maximum of $106$ text messages monthly.
\item[c)] Bob's total budget is $500+15h$ chf where $h$ is the number of hours he babysits. Bob's costs are $350+375+60\cdot 3=905$ chf. In order for Bob to afford the trip, he needs his budget to be greater or equal than his costs, i.e. the inequality $500+15h\geq 905$ has to hold. Thus, for Bob to figure out how much he has to babysit, he needs to solve
\begin{equation*}
\begin{split}
& 500+15h\geq 905 \\
& 15h\geq 405\\
&h\geq 27.
\end{split}
\end{equation*}
Therefore Bob has to babysit at least $27$ hours.
\end{sol}

\section{Functions}
A \underline{function} is a relationships between two sets $X$ (called the domain) and $Y$ (called the codomain) that assigns to each element $x\in X$ a unique element $y\in Y$. Function are most often denoted by letters such as $f$, $g$ and $h$. 

If $f$ if a function from the set $X$ to the set $Y$, we write $f:X\to Y$. Thus, let $f:X\to Y$ be a function. If $x$ is an element of $X$ ($x\in X$), then the element of $Y$ that is associated to $x$ by $f$ is denoted by $f(x)$ and is called the \underline{image of $x$ under $f$}. If $f(x)=y$ with $y\in Y$, then $x$ is called a \underline{preimage of $y$ in $X$}.

\textit{Example of functions}:
\begin{enumerate}
\item[$1$] Constant functions: $f:\mathbb{R}\to \mathbb{R}$, $f(x)=c$, where $c\in \mathbb{R}$ is a constant. \\
\underline{Note}: this is a general form to give constant functions. For example $g:\mathbb{N}\to \mathbb{N}$, $g(x)=2$ is a constant function. 
\item[$2$] Linear functions: $f:\mathbb{R}\to \mathbb{R}$,  $f(x)=ax+b$ where $a, b\in \mathbb{R}$ and $a\neq 0$. 
\item[$3$] Quadratic functions: $f:\mathbb{R}\to \mathbb{R}$,  $f(x)=ax^{2}+bx+c$ where $a, b, c\in \mathbb{R}$ and $a\neq 0$. 
\item[$4$] [Advanced example]: Constant, linear and quadratic functions are particular examples of a bigger family of functions called polynomial functions. These take the form: $f:\mathbb{R}\to \mathbb{R}$,  $f(x)=a_{n}x^{n}+a_{n-1}x^{n-1}+\cdots +a_{2}x^{2}+a_{1}x+a_{0}$ where $a_{n}, a_{n-1}, \dots, a_{1}, a_{0}\in \mathbb{R}$ and $a_{n}\neq 0$. See that when $n=0,1$ and $2$, respectively, $f$ is a constant, linear and quadratic, respectively, function.
\end{enumerate}

When we have a function  $f:X\to Y$, we sometimes write $y=f(x)$. In this case, we say that $x$ is the \underline{independent variable} of the function $f$, while $y$ is the \underline{dependent variable} of the function $f$.

\subsection{Evaluating functions}
Let $f:\mathbb{R}\to \mathbb{R}$ be a function. To \underline{evaluate} $f$ at a certain input value $c\in \mathbb{R}$ means that in the expression of $f$ we replace all instances of the variable with the input value $c$ and perform the calculation. This is denoted by $f(c)$. 

\textit{Example}: Let $f:\mathbb{R}\to \mathbb{R}$ be given by $f(x)=2x$. If we want to evaluate the function in the number $\textcolor{green}{2}$, we simply substitute the value $\textcolor{green}{2}$ for $x$ in the expression of $f(x)$: $f(\textcolor{green}{2})=2\cdot\textcolor{green}{ 2}=4$. We do the same if we want to evaluate $f$ in $\textcolor{yellow}{0}$, $\textcolor{orange}{\pi}$, $\textcolor{red}{-\frac{31}{45}}$, or any value $\textcolor{pink}{c}$ in our domain:
\begin{equation*}
\begin{split}
& f(\textcolor{yellow}{0})=2\cdot\textcolor{yellow}{0}=0\\
& f(\textcolor{orange}{\pi})=2\cdot\textcolor{orange}{\pi}=2\pi\\
& f(\textcolor{red}{-\frac{31}{45}})=2\cdot\textcolor{red}{-\frac{31}{45}}=-\frac{62}{45}\\
& f(\textcolor{pink}{c})=2\cdot\textcolor{pink}{c}=2c.
\end{split}
\end{equation*}

\begin{ex}\label{funct1}
For each of the following functions $f$ evaluate $f(x)$ in $2$, $0$, $\pi$, $-\frac{31}{45}$ and $c$ a random element in the domain of $f$.
\begin{enumerate}
\item[a)] $f:\mathbb{R}\to\mathbb{R}$, $f(x)=2$.
\item[b)] $f:\mathbb{R}\to\mathbb{R}$, $f(x)=7x-13$.
\item[c)] $f:\mathbb{R}\to\mathbb{R}$, $f(x)=2x^{2}-14$.
\item[d)] $f:\mathbb{R}\to\mathbb{R}$, $f(x)=(x+2)(x-6)$.
\item[e)] $f:(\mathbb{R}\setminus\{9\})\to\mathbb{R}$, $f(x)=\frac{8x-1}{x-9}$.
\item[f)] $f:[-4,\infty)\to\mathbb{R}$, $f(x)=\sqrt{x+4}$.
\item[g)] $f:\mathbb{R}\to\mathbb{R}$, $f(x)=3x^{3}+2x^{2}+x+1$.
\item[h)] $f:[-4,\infty)\to\mathbb{R}$, $f(x)=5x^{2}-\sqrt{2x+8}$.
\end{enumerate}
By $\mathbb{R}\setminus\{9\}$ we understand the set $\mathbb{R}$ from which we have removed the element $9$. When we write $x\in \mathbb{R}\setminus\{9\}$ it means that $x$ can be any real number except for $9$.
\end{ex}

\begin{sol}
\begin{enumerate}
\item[a)] We note that $f$ is a constant function. This means that for any input value $c\in \mathbb{R}$, the value of $f(c)$ remains constant and equal to $2$, i.e. $f(c)=2$ for all $c\in \mathbb{R}$. In particular, we have $f(2)=f(0)=f(\pi)=f(-\frac{31}{45})=2$.
\item[b)] We first do the particular evaluations:
\begin{equation*}
\begin{split}
& f(2)=7\cdot 2-13=14-13=1;\\
& f(0)=7\cdot 0-13=-13;\\
& f(\pi)=7\cdot \pi-13;\\
& f(-\frac{31}{45})=7\cdot (-\frac{31}{45})-13=-\frac{217}{45}-13=-\frac{802}{45}.
\end{split}
\end{equation*}
Lastly, for any $c\in \mathbb{R}$, we have $f(c)=7c-13$.
\item[c)] We first do the particular evaluations:
\begin{equation*}
\begin{split}
& f(2)=2\cdot 2^{2}-14=8-14=-6;\\
& f(0)=2\cdot 0^{2}-14=-14;\\
& f(\pi)=2\cdot \pi^{2}-14;\\
& f(-\frac{31}{45})=2\cdot (-\frac{31}{45})^{2}-14=\frac{1922}{2025}-14=-\frac{26428}{2025}.
\end{split}
\end{equation*}
Lastly, for any $c\in \mathbb{R}$, we have $f(c)=2c^{2}-14$.
\item[d)] We first do the particular evaluations:
\begin{equation*}
\begin{split}
& f(2)=(2+2)(2-6)=4\cdot (-4)=-16;\\
& f(0)=(0+2)(0-6)=2\cdot (-6)=-12;\\
& f(\pi)=(\pi+2)(\pi-6)=\pi^{2}+2\pi-6\pi-12=\pi^{2}-4\pi-12;\\
& f(-\frac{31}{45})=(-\frac{31}{45}+2)(-\frac{31}{45}-6)=\frac{59}{45}\cdot (-\frac{301}{45})=-\frac{17759}{2025}.
\end{split}
\end{equation*}
Lastly, for any $c\in \mathbb{R}$, we have $f(c)=(c+2)(c-6)$.
\item[e)] We first do the particular evaluations:
\begin{equation*}
\begin{split}
& f(2)=\frac{8\cdot 2-1}{2-9}=-\frac{15}{7};\\
& f(0)=\frac{8\cdot 0-1}{0-9}=\frac{1}{9};\\
& f(\pi)=\frac{8\pi-1}{\pi-9};\\
& f(-\frac{31}{45})=\frac{8\cdot (-\frac{31}{45})-1}{-\frac{31}{45}-9}=\frac{-\frac{293}{45}}{-\frac{436}{45}}=\frac{293}{436}.
\end{split}
\end{equation*}
Lastly, for any $c\in \mathbb{R}\setminus \{9\}$, we have $f(c)=\frac{8c-1}{c-9}$.
\item[f)] We first do the particular evaluations:
\begin{equation*}
\begin{split}
& f(2)=\sqrt{2+4}=\sqrt{6};\\
& f(0)=\sqrt{0+4}=\pm 2;\\
& f(\pi)=\sqrt{\pi+4};\\
& f(-\frac{31}{45})=\sqrt{-\frac{31}{45}+4}=\sqrt{\frac{149}{45}}.
\end{split}
\end{equation*}
Lastly, for any $c\in [-4,\infty)$, we have $f(c)=\sqrt{c+4}$.
\item[g)] We first do the particular evaluations:
\begin{equation*}
\begin{split}
& f(2)=3\cdot 2^{3}+2\cdot 2^{2}+2+1=24+8+3=35;\\
& f(0)=3\cdot 0^{3}+2\cdot 0^{2}+0+1=1;\\
& f(\pi)=3\pi^{3}+2\pi^{2}+\pi+1;\\
& f(-\frac{31}{45})=3\cdot (-\frac{31}{45})^{3}+2\cdot (-\frac{31}{45})^{2}-\frac{31}{45}+1=-\frac{89373}{91125}+\frac{1922}{2025}+\frac{14}{45}=\frac{25467}{91125}.
\end{split}
\end{equation*}
Lastly, for any $c\in \mathbb{R}$, we have $f(c)=3c^{3}+2c^{2}+c+1$.
\item[h)] We first do the particular evaluations:
\begin{equation*}
\begin{split}
& f(2)=5\cdot 2^{2}-\sqrt{2\cdot 2+8}=20-\sqrt{12}\\
& f(0)=5\cdot 0^{2}-\sqrt{2\cdot 0+8}=\sqrt{8};\\
& f(\pi)=5\pi^{2}-\sqrt{2\pi+8};\\
& f(-\frac{31}{45})=5\cdot (-\frac{31}{45})^{2}-\sqrt{2\cdot (-\frac{31}{45})+8}=\frac{961}{405}-\sqrt{\frac{298}{45}}.
\end{split}
\end{equation*}
Lastly, for any $c\in [-4,\infty)$, we have $f(c)=5c^{2}-\sqrt{2c+8}$.
\end{enumerate}
\end{sol}

\begin{ex}
For each function $f$ given in Exercise \ref{funct1} Items b) therough e), determine $x$ in the domain of $f$ (if it exists) such that $f(x)=0$, $f(x)=1$ and $f(x)=-1$.
\end{ex}

\begin{sol}
We begin with $f$ from Item b). We have $f:\mathbb{R}\to\mathbb{R}$ with $f(x)=7x-13$. To find $x\in \mathbb{R}$ such that $f(x)=0$, we need to solve the equation $7x-13=0$. The solution of this equation is $\frac{13}{7}$, and so we conclude that for $x=\frac{13}{7}$ we have $f(\frac{13}{7})=0$. Similarly one shows that for $x=2$ we have $f(2)=1$; and for $x=\frac{12}{7}$ we have $f(\frac{12}{7})=-1$.

We now take $f$ from Item c). We have $f:\mathbb{R}\to\mathbb{R}$ with $f(x)=2x^{2}-14$. To find $x\in \mathbb{R}$ such that $f(x)=0$, we need to solve the quadratic equation $2x^{2}-14=0$. We have:
\begin{equation*}
\begin{split}
&2x^{2}-14=0 \\
& 2x^{2}=14\\
& x^{2}=7\\
& x=\pm \sqrt{7}.
\end{split}
\end{equation*}
We conclude that for both $x=\sqrt{7}$ and $x=-\sqrt{7}$ we have $f(\sqrt{7})=f(-\sqrt{7})=0$. Similarly one shows that for both $x=\sqrt{\frac{15}{2}}$ and $x=-\sqrt{\frac{15}{2}}$ we have $f(\sqrt{\frac{15}{2}})=f(-\sqrt{\frac{15}{2}})=1$; while for both $x=\sqrt{\frac{13}{2}}$ and $x=-\sqrt{\frac{13}{2}}$ we have $f(\sqrt{\frac{13}{2}})=f(-\sqrt{\frac{13}{2}})=-1$.
 
Let $f$ be as in Item d): $f:\mathbb{R}\to\mathbb{R}$ with $f(x)=(x+2)(x-6)$. To find $x\in \mathbb{R}$ such that $f(x)=0$, we need to solve $(x+2)(x-6)=0$. Clearly, the solutions of this are $x=-2$ and $x=6$; and we have $f(-2)=f(6)=0$. Similarly, to find $x\in \mathbb{R}$ such that $f(x)=1$, we need to solve $(x+2)(x-6)=x^{2}-4x-12=1$. We have:
\begin{equation*}
\begin{split}
&x^{2}-4x-12=1\\
& x^{2}-4x+4-4=13\\
& (x-2)^{2}=17\\
& x-2=\pm\sqrt{17}.
\end{split}
\end{equation*}
Thus $x=2+\sqrt{17}$ and $x=2-\sqrt{17}$ are the solutions of $(x+2)(x-6)=1$, and so we have $f(2+\sqrt{17})=f(2-\sqrt{17})=1$. Similarly, one shows that for $x=2+\sqrt{15}$ and $x=2-\sqrt{15}$ we have $f(2+\sqrt{15})=f(2-\sqrt{15})=-1$. 

Lastly, let $f$ be as in Item e): $f:(\mathbb{R}\setminus\{9\})\to\mathbb{R}$ with $f(x)=\frac{8x-1}{x-9}$. To find $x\in \mathbb{R}\setminus\{9\}$, such that $f(x)=0$, we need to solve $\frac{8x-1}{x-9}=0$. This comes down to solving $8x-1=0$ which has the unique solution $x=\frac{1}{8}$. Thus, for $x=\frac{1}{8}$ we have $f(\frac{1}{8})=0$. To find $x\in \mathbb{R}\setminus\{9\}$, such that $f(x)=1$, we need to solve $\frac{8x-1}{x-9}=1$:
\begin{equation*}
\begin{split}
&\frac{8x-1}{x-9}=1\textcolor{blue}{\ | \cdot (x-9)}\\
& 8x-1=x-9\\
& 7x=-8\\
& x=-\frac{8}{7}.
\end{split}
\end{equation*}
Thus, for $x=-\frac{8}{7}$ we have $f(-\frac{8}{7})=1$. Similarly, one shows that for $x=\frac{10}{9}$ we have $f(\frac{10}{9})=-1$.
\end{sol}

\subsection{Domain, Codomain, Range of a Function}

\textit{Example}: Let $h:\mathbb{R}\to \mathbb{R}$, $h(x)=x^{2}-2$. Note that $h$ is indeed a function since given any input $x\in \mathbb{R}$, there is exactly one output $h(x)\in \mathbb{R}$. We evaluate $h$ in some values $x\in \mathbb{R}$:
\begin{center}
\begin{equation*}
\begin{split}
& h(-2)=2;\\
& h(5)=23;\\
& h(\sqrt{2})=0;\\
& h(-\sqrt{2})=0.
\end{split}
\end{equation*}
\end{center}
Notice that the number $0$ in the codomain has two preimages: $\sqrt{2}$ and $-\sqrt{2}$. This does not violate the mathematical definition of a function since the definition only states that each element of the domain has exactly one image in the codomain. The definition does not stipulate that two different inputs must produce different outputs.

Finding the preimages of an element in the codomain can sometimes be difficult. In general, if $y$ is in the codomain, to find its preimages, we need to ask, “For which values of $x$ in the domain will we have $y=h(x)$?” 

\textit{Example}: For the function $h$, let us determine the preimages of $5$ in its domain. For this, we need to find $x\in \mathbb{R}$ such that $h(x)=5$, i.e. we need to find $x\in \mathbb{R}$ such that $x^{2}-2=5$. We solve the quadratic equation and see that the preimages of $5$ in $\mathbb{R}$ are $\sqrt{7}$ and $-\sqrt{7}$.

\begin{ex}
Notice that for the function $h$ not every element in the codomain has a preimage. Show that there is no $x\in \mathbb{R}$ such that $h(x)=-3$. \\
(Advanced): Show that any number $y\in [-2, \infty)$ has (at least) one preimage in $\mathbb{R}$.
\end{ex}

\begin{sol}
Assume there exists $x\in \mathbb{R}$ such that $h(x)=-3$. Then, we have:
\begin{equation*}
\begin{split}
& x^{2}-2=-3\\
& x^{2}=-5.\\
\end{split}
\end{equation*}
However, the equation $x^{2}=-5$ has no solution in $\mathbb{R}$, as for any $c\in \mathbb{R}$ we have $c^{2}\geq 0$. We conclude that $-3$ has no preimage in $\mathbb{R}$.

We now show that any $y\in [-2, \infty)$ has (at least) one preimage in $\mathbb{R}$. Let $y\in [-2, \infty)$. To show that $y$ has a preimage $x\in \mathbb{R}$, means to show that the equation $x^{2}-2=y$ has a solution in $\mathbb{R}$. We rewrite and have $x^{2}=y+2$. Now, as $y\geq -2$, it follows that $y+2\geq 0$. Thus, we see that $\sqrt{y+2}$ is a solution of $x^{2}-2=y$: $(\sqrt{y+2})^{2}-2=y+2-2=y$. This shows that any $y\in [-2, \infty)$ has a preimage $\sqrt{y+2}\in \mathbb{R}$.
\end{sol}

\begin{ex}
Let $f:\mathbb{R}\to \mathbb{R}$ be defined by $f(x)=x^{2}-5x$ and let $g:\mathbb{Z}\to \mathbb{Z}$ be defined by  $g(m)=m^{2}-5m$.
\begin{enumerate}
\item[a)] Determine $f(-3)$ and $f(\sqrt{8})$.
\item[b)] Determine $g(2)$ and $g(-2)$.
\item[c)] Determine the set of all preimages of $6$ for the function $f$. 
\item[d)] Determine the set of all preimages of $6$ for the function $g$. 
\item[e)] Determine the set of all preimages of $2$ for the function $f$. 
\item[f)] Determine the set of all preimages of $2$ for the function $6$. 
\end{enumerate}
\end{ex}

\begin{sol}
\begin{enumerate}
\item[a)] We have $f(-3)=(-3)^{2}-5\cdot (-3)=9+15=24$ and $f(\sqrt{8})=(\sqrt{8})^{2}-5\cdot \sqrt{8}=8-5\sqrt{8}$..
\item[b)] We have $g(2)=2^{2}-5\cdot 2=4-10=-6$ and $g(-2)=(-2)^{2}-5\cdot (-2)=4+10=14$.
\item[c)] We need to find the solutions $x\in \mathbb{R}$ of the equation: $x^{2}-5x=6$. We complete the square and we have:
 $$x^{2}-5x-6=x^{2}-2\cdot \frac{5}{2}x+\frac{25}{4}-\frac{25}{4}-6=(x-\frac{5}{2})^{2}-\frac{49}{4}.$$
We arrived at the equation: $(x-\frac{5}{2})^{2}-\frac{49}{4}=0$. We now solve:
\begin{equation*}
\begin{split}
& (x-\frac{5}{2})^{2}-\frac{49}{4}=0\\
& (x-\frac{5}{2})^{2}=\frac{49}{4}.\\
& x-\frac{5}{2}=\pm \frac{7}{2}.
\end{split}
\end{equation*}
If $x-\frac{5}{2}= \frac{7}{2}$, then $x=3$ and, if $x- \frac{5}{2}=- \frac{7}{2}$, then $x=-1$. Thus, the preimages of $6$ in $\mathbb{R}$ are $3$ and $-1$.
\item[d)] We need to find the solutions $m\in \mathbb{Z}$ of the equation: $m^{2}-5m=6$. In Item c) we have seen that the solutions of the equation $m^{2}-5m=6$ in $\mathbb{R}$ are $3$ and $-1$. As both $3$ and $-1$ are elements of $\mathbb{Z}$, we conclude that the preimages of $6$ are $3$ and $-1$.
\item[e)] We need to find the solutions $x\in \mathbb{R}$ of the equation: $x^{2}-5x=2$. We complete the square and we have:
 $$x^{2}-5x-2=x^{2}-2\cdot \frac{5}{2}x+\frac{25}{4}-\frac{25}{4}-2=(x-\frac{5}{2})^{2}-\frac{33}{4}.$$
We arrived at the equation: $(x-\frac{5}{2})^{2}-\frac{33}{4}=0$. We now solve:
\begin{equation*}
\begin{split}
& (x-\frac{5}{2})^{2}-\frac{33}{4}=0\\
& (x-\frac{5}{2})^{2}=\frac{33}{4}.\\
& x-\frac{5}{2}=\pm \frac{\sqrt{33}}{2}.
\end{split}
\end{equation*}
If $x-\frac{5}{2}= \frac{\sqrt{33}}{2}$, then $x=\frac{5+\sqrt{33}}{2}$ and, if $x- \frac{5}{2}=- \frac{\sqrt{33}}{2}$, then $x=\frac{5-\sqrt{33}}{2}$. Thus, the preimages of $2$ in $\mathbb{R}$ are $\frac{5+\sqrt{33}}{2}$ and $\frac{5-\sqrt{33}}{2}$.
\item[f)] We need to find the solutions $m\in \mathbb{Z}$ of the equation: $m^{2}-5m=2$. In Item e) we have seen that the solutions of the equation $m^{2}-5m=2$ in $\mathbb{R}$ are $\frac{5+\sqrt{33}}{2}$ and $\frac{5-\sqrt{33}}{2}$. As neither $\frac{5+\sqrt{33}}{2}$ nor $\frac{5-\sqrt{33}}{2}$ are elements of $\mathbb{Z}$, we conclude that the $2$ doesn't have preimages in $\mathbb{Z}$.
\end{enumerate}
\end{sol}

Let  $f:X\to Y$. The set  $\{f(x)\mid x\in X\}$ is called the \underline{range} or \underline{image} of the function $f$. We denote it by $\range(f)$. Think of the range of $f$ as the subset of the codomain 
$Y$ containing those elements $y\in Y$ which have at least one preimage in $X$ i.e. the elements $y\in Y$ for which there exists $x\in X$ such that $f(x)=y$. 

\textit{Example}: For the function $h:\mathbb{R}\to \mathbb{R}$, $h(x)=x^{2}-2$, we have seen that $2, 23, 0\in \range(h)$. However, the number $-3$ (which is an element of the codomain) is not an element of $\range(h)$ as we have shown that there does not exist $x\in \mathbb{R}$ such that $h(x)=-3$.

The range of the function $f$ can be equivalently defined as $\range(f)=\{y\in Y\mid \ y=f(x) \text{ for some }x\in X\}$. Notice that this means that $\range(f)$ is a subset of $Y$, but does not necessarily mean that $\range(f)=Y$. Whether we have this set equality or not depends on the function $f$.

\begin{ex}
Let  $b$ be the function that assigns to each person his or her birthday (month and day).
\begin{enumerate}
\item[a)]  What is the domain of this function?
\item[b)] What is a codomain for this function?
\item[c)] Assume that for each day of the year there exists a person with birthday on that day. What does this tell us about the range of the function $b$?
\end{enumerate}
\end{ex}

\begin{sol}
\begin{enumerate}
\item[a)] The domain of $b$ is the set of all people. 
\item[b)] The codomain of $b$ is the set of all dates in the calendar. 
\item[c)] The range of $b$ is the same as the codomain of $b$, as for each date there is a person with birthday on that date.
\end{enumerate}
\end{sol}

\subsection{The Graph of a Function}

A function $f$ is uniquely represented by the set of all pairs $(x,f(x))$, called the \underline{graph of the} \underline{function}. When the domain and the codomain are sets of real numbers, we can think of each such pair as a point in the Cartesian plane.

\begin{ex}
Let $f:\mathbb{R}\to \mathbb{R}$ be defined by $f(x)=x^{2}-2x$.
\begin{enumerate}
\item[a)]  Evaluate $f(-3), f(-1), f(1)$ and $f(3)$.
\item[b)] Determine all the preimages of $0$ and all the preimages of $4$.
\item[c)] Sketch the graph of the function $f$.
\item[d)] Determine the range of the function $f$.
\end{enumerate}
\end{ex}

\begin{sol}
\begin{enumerate}
\item[a)]  We have 
\begin{equation*}
\begin{split}
& f(-3)=(-3)^{2}-2\cdot (-3)=9+6=15.\\
& f(-1)=(-1)^{2}-2\cdot (-1)=1+2=3.\\
& f(1)=1^{2}-2\cdot 1=1-2=-1.\\
& f(3)=3^{2}-2\cdot 3=9-6=3.
\end{split}
\end{equation*}
\item[b)] To determine all the preimages of $0$ we need to solve the equation $x^{2}-2x=0$. We see that $x^{2}-2x=x(x-2)$ and so the solutions of $x(x-2)=0$ are $0$ and $2$. Thus, the preimages of $0$ are $0$ and $2$.

To determine all the preimages of $4$ we need to solve the equation $x^{2}-2x=4$. We complete the square and then solve the resulting equation:
$$x^{2}-2x-4=x^{2}-2x+1-1-4=(x-1)^{2}-5$$
and 
\begin{equation*}
\begin{split}
& (x-1)^{2}-5=0\\
&(x-1)^{2}=5\\
& x-1=\pm \sqrt{5}.
\end{split}
\end{equation*}
If $x-1=\sqrt{5}$, then $x=1+\sqrt{5}$ and, if $x-1=-\sqrt{5}$, then $x=1-\sqrt{5}$. We have determined that the preimages of $4$ are $1+\sqrt{5}$ and $1-\sqrt{5}$
\item[c)] We see that $f$ is a quadratic function. Thus, to draw its graph it suffices to know the vertex point, the shape of the curve and several points through which the graph passes.

To determine the vertex, we write $f$ in its vertex form by completing the square:
$$f(x)=x^{2}-2x=x^{2}-2x+1-1=(x-1)^{2}-1.$$
Thus, the vertex point of $f$ is $(1,-1)$. As the leading coefficient of $f$ (the coefficient of $x^{2}$) is greater than $0$, we know that the vertex is a minimum and that the graph of $f$ opens upward.  

We have seen in Item b) that the zeroes of $f$ are $0$ and $2$. Thus, we know that the graph passes through the points $(0,0)$ and $(2,0)$. We have also seen in Item a) that the points $(-3,15), \ (-1,3)$ and $(3,3)$ belong to the graph of $f$.

We note that as the vertex of $f$ is in $(1,-1)$, the line given by $x=1$ is the axis of symmetry of the graph of $f$. This means that $f(1-x)=f(x+1)$ for all $x\in \mathbb{R}$. We can actually manually check that $f(1-x)=f(x+1)$ holds for all $x\in \mathbb{R}$. On one hand we have:
$$f(1-x)=(1-x)^{2}-2(1-x)=1-2x+x^{2}-2+2x=x^{2}-1$$
and on the other we have:
$$f(x+1)=(x+1)^{2}-2(x+1)=1+2x+x^{2}-2-2x=x^{2}-1.$$
Thus $f(1-x)=f(x+1)$ holds for all $x\in \mathbb{R}$. With this property in mind, we determine pairs of points belonging to the graph of $f$:\\
Since the point $(-1,3)$ belongs to the graph of $f$ (see Item a)), then its symmetric with respect to the axis of symmetry $x=1$ belongs to the graph, i.e. the point $(2,3)$ belongs to the graph of $f$.\\
\underline{Note}: The abscissa of the point $(0,0)$ of the graph of $f$ is situated at distance $1$ from the abscissa of the vertex point $(1,-1)$ on the $x$-axis. Similarly, the abscissa of the point $(-1,3)$ of the graph of $f$ is situated at distance $2$ from the abscissa of the vertex point $(1,-1)$ on the $x$-axis. To determine other points on the graph we take points whose abscissa is at a distance we have not yet considered from the abscissa of the vertex, and calculate its ordinate:\\
$-\frac{1}{2}$ is at a distance $\frac{3}{2}$ from the abscissa of the vertex and we have:
$$f(-\frac{1}{2})=(-\frac{1}{2})^{2}-2(-\frac{1}{2})=\frac{1}{4}+1=\frac{3}{4}.$$
Thus the point $(-\frac{1}{2}, \frac{3}{4})$ belongs to the graph of $f$. Moreover, by the symmetry of the graph, the point $(\frac{5}{2}, \frac{3}{4})$ also belongs to the graph of $f$. \\

We can now draw the graph:
\begin{center}
\resizebox{6cm}{!}{\begin{tikzpicture}
    % Building main structure
    % xlabel, ylabel, xmin, xmax, ymin, ymax
    \drawcustomgrid{x}{f(x)}{-5}{5}{-5}{10}
	\clip (-5,-5) rectangle (5,10);
    \draw[thick, smooth, red] plot[domain=-5:5, samples=100] (\x, {\x*\x - 2*\x});
    \foreach \Point in {(0,0), (2,0), (1,-1), (3,3), (-1,3)}{
    \node at \Point {$\circ$};
    }
\end{tikzpicture}}
\end{center}
\item[d)] We see from the graph of $f$ that $\range(f)=[-1,\infty)$.

Another way to argue this is to note that the vertex of $f$ is in the point $(1,-1)$ and that $f$ opens upwards (as the leading coefficient of $f$ is positive). Thus, we have $\range(f)=[-1,\infty)$.
\end{enumerate}
\end{sol}

\begin{ex}
Let $f:\mathbb{Z}\to \mathbb{Z}$ be defined by $f(m)=2m+1$.
\begin{enumerate}
\item[a)]  Evaluate $f(-7), f(-3), f(3)$ and $f(7)$.
\item[b)] Determine all the preimages of $5$ and all the preimages of $4$.
\item[c)] Sketch the graph of the function $f$.
\item[d)] Determine the range of the function $f$.
\end{enumerate}
\end{ex}

\begin{sol}
\begin{enumerate}
\item[a)]  We have 
\begin{equation*}
\begin{split}
& f(-7)=2\cdot (-7)+1=-14+1=-13.\\
& f(-3)=2\cdot (-3)+1=-6+1=-5.\\
& f(3)=2\cdot 3+1=6+1=7.\\
& f(7)=2\cdot 7+1=14+1=15.
\end{split}
\end{equation*}
\item[b)] To determine all the preimages of $5$ in $\mathbb{Z}$ we need to solve the equation $2m+1=5$ in $\mathbb{Z}$. We get $m=2$, and so $2$ is the preimage of $5$. \\
\underline{Recall}: The solution of a linear equation in one variable is unique. Thus any $y\in \mathbb{Z}$ can have at most one preimage in $\mathbb{Z}$ under $f$.

To determine all the preimages of $4$ in $\mathbb{Z}$ we need to solve the equation $2m+1=4$ in $\mathbb{Z}$. We get $m=\frac{3}{2}$. But $\frac{3}{2}\notin \mathbb{Z}$, and so we conclude that $4$ has no preimage in $\mathbb{Z}$. 
\item[c)] Because the domain of $f$ in $\mathbb{Z}$, the graph of $f$ will consist of the collection of points $(m,2m+1)$ for $m\in \mathbb{Z}$. Remark that this collection of points belongs to the line given by $y=2x+1$.

We have seen in Item a) that the points $(-7,-13), (-3,-5), (3,7)$ and $(7,15)$ belong to the graph of $f$. We determine some more points belonging to the graph of $f$:
\begin{equation*}
\begin{split}
& f(-4)=2\cdot (-4)+1=-8+1=-7.\\
& f(-2)=2\cdot (-2)+1=-4+1=-3.\\
& f(-1)=2\cdot (-1)+1=-2+1=-1.\\
& f(1)=2\cdot 1+1=2+1=3.\\
& f(2)=2\cdot 2+1=4+1=5.\\
& f(4)=2\cdot 4+1=8+1=9.
\end{split}
\end{equation*}

\begin{center}
\begin{tikzpicture}[x=1cm,y=0.4cm]
  \drawcustomgrid{m}{f(m)}{-5}{5}{-10}{10}
	\clip (-5,-10) rectangle (5,10);

\foreach \Point in {(-3,-5), (3,7), (-2,-3), (-1,-1), (1,3), (2,5), (-4,-7), (4,9)}{
    \node at \Point {\textbullet};
}
\end{tikzpicture}
\end{center}
\item[d)] Let $y\in \range(f)$. Then, there exists some $n\in \mathbb{Z}$ such that $y=2n+1$. Thus, $y$ is an odd integer. It follows that $\range(f)$ is a subset of the set of odd integers, i.e. $\range(f)\subseteq \mathcal{O}$, where by $\mathcal{O}$ we have denoted the set of odd integers.

Now let $d\in \mathbb{Z}$ be an odd integer. Then we can write $d$ as $d=2d_{1}+1$ where $d_{1}\in \mathbb{Z}$. It follows that $d=f(d_{1})$, and thus $d\in \range(f)$. This shows that the set of odd integers is a subset of $\range(f)$, i.e. $\mathcal{O}\subseteq \range(f)$. Lastly, as $\range(f)\subseteq \mathcal{O}$ and $\mathcal{O}\subseteq \range(f)$, we conclude that $\range(f)=\mathcal{O}$
\end{enumerate}
\end{sol}

\end{document}